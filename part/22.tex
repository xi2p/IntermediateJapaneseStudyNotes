\chapter{標準日本語 第22課}

\separatorline

\section{文法}

\subsection{「はもとより」表不仅・・・而且・・・}

% 语法结构
\begin{codeblock}
・・・はもとより、 ~
\end{codeblock}

% 语法解释
表示“不仅・・・,而且・・・也~”,\textbf{“不用说・・・,连・・・都~”}。用于书面语。

\textbf{后句要接「まで」「も」等表示程度的词语。}

% 语法示例
\begin{example}
    \begin{itemize}
        \item 友達\textbf{はもとより}、両親\textbf{まで}私の意見に反対した。
        
        (不仅朋友反对,连父母也反对我的意见。)

        \item このデパートは、休日\textbf{はもとより}、平日\textbf{まで}混雑する。
        \item 家の中\textbf{はもとより}、外で\textbf{も}使用できます。
    \end{itemize}
\end{example}

\separatorline
\subsection{「にしろ」表也罢}

% 语法结构
\begin{codeblock}
・・・にしろ、 ・・・にしろ、 ~
\end{codeblock}

% 语法解释
表示\textbf{“也罢・・・,也罢・・・,都~”。}用于列举两种或多种情况,表示无论哪种情况,结果都是一样的。用于书面语。

「\textbf{にしろ」前面可以接动词普通体,形容词普通体,名词,形容动词词干。}

% 语法示例
\begin{example}
    \begin{itemize}
        \item 肉\textbf{にしろ}魚\textbf{にしろ}、新鮮なものは美味しい。
        \item 図書館\textbf{にしろ}美術館\textbf{にしろ}、文化施設が近くにあるのはいい。
        \item 行く\textbf{にしろ}行かない\textbf{にしろ}、一度電話をかけてください。
        \item 暑い\textbf{にしろ}寒い\textbf{にしろ}、外に出ないのだから関係ない。
    \end{itemize}
\end{example}

可以用「せよ」代替「しろ」,显得更正式一些。

\begin{example}
    \begin{itemize}
        \item コーヒー\textbf{にせよ}、紅茶\textbf{にせよ}、飲み過ぎてはいけない。
    \end{itemize}
\end{example}

\separatorline

\subsection{「とばかりに」表仿佛要说}

% 语法结构
\begin{codeblock}
・・・とばかりに、 ~
\end{codeblock}

% 语法解释
表示\textbf{“仿佛要说・・・”},“好像在说・・・一样”,\textbf{用于描述某人的表情或动作},给人一种好像在说某句话的感觉。用于书面语。

\begin{callout}
    此处的「ばかり」表示极限程度,“几乎就要・・・”。
\end{callout}

\textbf{「と」前面可以接动词普通体,动词命令形,或引用的句子。}

% 语法示例
\begin{example}
    \begin{itemize}
        \item 失敗したのは私のせいだ\textbf{とばかりに}、彼は私を叱った。
        
        (他仿佛要说“失败是我的错”一样,责备了我。)

        \item 彼に続け\textbf{とばかりに}、男たちは川へ飛び込んだ
        
        (男人们仿佛在喊“跟上!”一样,跳进了河里。)

        \item 「子供は邪魔だ。」\textbf{とばかりに}、追い返された。
    \end{itemize}
\end{example}

\separatorline

\subsection{「と言えばそれまでだが」表虽说・・・但也不过如此}

% 语法结构
\begin{codeblock}
・・・と言えばそれまでだが、 ~
\end{codeblock}

% 语法解释
表示“如果那么说的话,事情也就到此为止、无需多谈了,但是~”。
\textbf{先姑且承认一个表面的,简单的,或会导致讨论终结的理由或说法,然后话锋一转,提出自己更深入、不同的看法。}
核心在于“让步后的转折论述”,是一种辩论或深入分析的常用句式。


% 语法示例
\begin{example}
    \begin{itemize}
        \item 彼は悪い\textbf{と言えばそれまでですが}、他の人にも責任があります。
        
        (说是他不好,那确实是不好,但其他人也有责任。)

        \item 実力が違う\textbf{と言えばそれまでだが}、あまりにもひどい負け方だ。
        
        (可以说是实力不同,但输得也太惨了。)
    \end{itemize}
\end{example}

\separatorline

\subsection{「というところだ」表观点陈述}

% 语法结构
\begin{codeblock}
・・・というところだ

・・・というところだろう

・・・というところであろう
\end{codeblock}

% 语法解释
\textbf{说话人虽然不能明确给出结论,但给出一个自认为大体上的妥当的观点。}相当于“大致上是・・・”。\textbf{这里的观点通常是对现状、程度的总结。}

% 语法示例
\begin{example}
    \begin{itemize}
        \item 事故の原因は、恐らく誰かの不注意\textbf{というところだろう}。
        
        (事故的原因,大概是某人的不注意。)

        \item あれほど練習したのに負けたなんて、実力が違う\textbf{というところであろう。}
    \end{itemize}
\end{example}


\begin{callout}
    「というところだ」用于归纳总结现状、程度,强调“这是我的看法”。

    \vspace{10pt}

    「というものだ」阐述普遍道理、事物的本质,带有说教、感慨、认定的语气。

    \vspace{10pt}

    「ということだ」传达、解释信息,最客观。
\end{callout}
