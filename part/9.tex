\chapter{標準日本語 第9課}

摘要:\textbf{無し}

\separatorline

\section{文法}

\subsection{「なんて」}

% 语法结构
\begin{codeblock}
・・・なんて、 ~
\end{codeblock}

% 语法解释
「なんて」可以表达多种意思。其表达十有八九都带有\textbf{“轻视”}或\textbf{“意外”}的感情色彩,且多用于\textbf{口语}中。

\subsubsection{评价出乎意料的事物}
「なんて」前面接令人\textbf{出乎意料}的事情,后面接对此的\textbf{评价}。

作此用法时,「なんて」前面接动词/一类形容词/形容动词/名词的\textbf{普通体}。

% 语法示例
\begin{example}
    \begin{itemize}
        \item 雨が\textbf{降るなんて}、困りましたね。
        
        \item 彼女はクラスで背が一番\textbf{高いなんて}、驚いてしまう。
        
        \item こんなに高い値段\textbf{だなんて}、信じられない。
        
        \item 君がこの会社の社員\textbf{ではなかったなんて}、知らなかった。
        
        (你之前不是这家公司的员工这件事,我还不知道呢。)
    \end{itemize}
\end{example}

\subsubsection{表示发言内容}

相当于「~などと」,\textbf{表示说话的内容}。前接短句。

\begin{example}
    \begin{itemize}
        \item 旅行が好きだ\textbf{なんて}、嘘を言ってしまった。
        
        (喜欢旅行这种事,真是说谎了。)
    \end{itemize}
\end{example}

\subsubsection{表示轻视的内容}

「なんて」前接\textbf{轻视}的内容。前面接动词/一类形容词/形容动词的普通体,或\textbf{直接接名词}。

\begin{example}
    \begin{itemize}
        \item \textbf{日本料理なんて}、嫌い。
        \item \textbf{殴るなんて}、ひどい。
    \end{itemize}
\end{example}

\subsubsection{表示列举的内容}

表示\textbf{列举},前面接名词。语义较为中性。

\begin{example}
    \begin{itemize}
        \item ジョン\textbf{なんて}名前はよくある。
        
        (\textbf{像}约翰\textbf{这种}名字很常见。)
    \end{itemize}
\end{example}

\subsubsection{修饰后接名词}

可以在「なんて」后面直接接名词,对其进行修饰,表示轻视、意外、列举。

\begin{example}
    \begin{itemize}
        \item 田中\textbf{なんて}人、知りません。(轻视)

        \item \textbf{漫画なんて}ものは、読んだことがあります。(例举)
        
        \item 彼女が\textbf{泣くなんて}ことは珍しい。(意外)

    \end{itemize}
\end{example}



\separatorline
\subsection{一类形容词+「まる / める」}

% 语法结构
\begin{codeblock}
一类形容词 +「まる / める」
\end{codeblock}

% 语法解释
一类形容词 +「まる」可以把形容词变成\textbf{自动词},表示“变得~”。

% 语法示例
\begin{example}
    \begin{itemize}
        \item 風が\textbf{弱まった}ら、出かけましょう。(\textbf{风小了}的话,我们就出发吧。)
        
        \item 気持ちが\textbf{高まる}。
    \end{itemize}
\end{example}

一类形容词 +「める」可以把形容词变成\textbf{他动词},表示“使~”。

\begin{example}
    \begin{itemize}
        \item 火をもっと\textbf{強めて}ください。(请\textbf{把火加大}一些。)
        
        \item 土を\textbf{固めて}家を作る。
    \end{itemize}
\end{example}

\separatorline

\subsection{「つつある」表变化正在进行}

% 语法结构
\begin{codeblock}
[・・・动词连用形] つつある
\end{codeblock}

% 语法解释
表示\textbf{变化正在进行},类似于「~ている」,但明确标明\textbf{变化}正在进行,更正式,多用于\textbf{书面语}。

「つつある」前面接\textbf{动词连用形}。

% 语法示例
\begin{example}
    \begin{itemize}
        \item 街の様子は\textbf{変わりつつある}。
        \item 弟も、だんだん\textbf{成長しつつある}。
        \item この川の水は、年々汚く\textbf{なりつつある}。(这条河的水,年年都在\textbf{变得}越来越脏。)
        
    \end{itemize}
\end{example}
