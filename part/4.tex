\chapter{標準日本語 第4課}

摘要:\textbf{無し}

\separatorline

\section{文法}

\subsection{「わりには」表不符合预期}

% 语法结构
\begin{codeblock}
・・・わりには、 ~
\end{codeblock}

% 语法解释
表示后句与前句的内容相比,\textbf{不符合预期}。相当于\textbf{“明明・・・,却~”}。类似于「・・・のに、~」,更委婉。

「わり」可以视作名词,其\textbf{前接续规则与名词一致}。(动词/一类形普通体、形容动词+な、名词+の)

语法中的「は」是用于强调,可以省略。

% 语法示例
\begin{example}
    \begin{itemize}
        \item 姉はよく\textbf{食べるわりには}、太らない。
        
        (姐姐\textbf{明明}吃得很多,\textbf{却}不胖。)

        \item このレストランの料理は\textbf{高いわりに}、あまり美味しくない。
        \item 彼は言葉が\textbf{丁寧なわりには}、態度が乱暴だ。
        \item この靴は、\textbf{値段のわりに}、丈夫だ。
        
        (这双鞋\textbf{明明}价格不贵,\textbf{却}很结实。)
    \end{itemize}
\end{example}

\separatorline

\subsection{「の(ん)だから」表原因、建议}

% 语法结构
\begin{codeblock}
・・・のだから、 ~

・・・んだから、 ~
\end{codeblock}

% 语法解释
\textbf{类似于「・・・から、~」},利用「の(ん)」进行强调。\textbf{后句常用于提出建议、命令、请求等}。

此语法常用于口语,尤其是「んだから」,仅用于口语。

% 语法示例
\begin{example}
    \begin{itemize}
        \item 夜に\textbf{出かけるんだから}、母が心配するのはあたりまえだ。
        
        (\textbf{因为}要晚上出门,妈妈担心是理所当然的。)

        \item 彼女はあなたよりも年が\textbf{下なんだから}、もっと親切にしてあげなさい。
    \end{itemize}
\end{example}

\separatorline

\subsection{「うえに」表再加上}

% 语法结构
\begin{codeblock}
・・・うえに、 ~
\end{codeblock}

% 语法解释
表示更进一层,相当于“・・・,再加上~”。类似于英文\textbf{"in addition to"}。

「うえ」是「上」,名词性质,\textbf{前接续规则与名词一致}。(动词/一类形普通体、形容动词+な、名词+の)

% 语法示例
\begin{example}
    \begin{itemize}
        \item 純子さんは\textbf{遅刻するうえに}、宿題も忘れた。
        
        (纯子\textbf{不仅}迟到,\textbf{而且}还忘记带作业。)

        \item この八百屋の野菜は、\textbf{新鮮なうえに}、値段が安い。
    \end{itemize}
\end{example}

\separatorline

\subsection{「たらいい / ばいい」表建议}

% 语法结构
\begin{codeblock}
・・・たらいい

・・・ばいい
\end{codeblock}

% 语法解释
类似于「といい」,\textbf{表示对某事的建议或希望}。可以把「いい」替换为「よろしい」,使语气更为礼貌。

「・・・ばよろしいでしょうか」常用于询问对方是否合适。

% 语法示例
\begin{example}
    \begin{itemize}
        \item 話したいことがあったら、\textbf{話したらいい}。
        
        \item 疲れたから、隣の部屋で\textbf{寝ればいい}。
        
        \item 明日何時ごろ\textbf{伺えばよろしいでしょうか}。
        
        (明天大约什么时候去拜访比较好呢?)
    \end{itemize}
\end{example}

\separatorline

\subsection{「~でもあり、 また~でもある」表有双重性质}

% 语法结构
\begin{codeblock}
~でもあり、 また~でもある
\end{codeblock}

% 语法解释
表示某事物\textbf{既有A的性质,又有B的性质}。

相当于“既是~,又是~” “既有~的一面,也有~的一面”。

\textbf{「~」处可以是形容动词词干、名词。若要使用一类形容词,需采用「词干+く+もあり」的形式。}

% 语法示例
\begin{example}
    \begin{itemize}
        \item 外国旅行は、楽\textbf{でもあり}、\textbf{また}不安\textbf{でもある。}
        \item 文章を書くことは、私の仕事\textbf{でもあり}、\textbf{また}趣味\textbf{でもあります。}
        \item 私の先生は、厳し\textbf{くもあり}、\textbf{また}優し\textbf{くもあります。}
    \end{itemize}
\end{example}

\separatorline

\subsection{「ところに」表动作的发生时间点}

% 语法结构
\begin{codeblock}
・・・ところに、 ~
\end{codeblock}

% 语法解释
此处「ところ」的含义和此前学过的「・・・ところです」相同,\textbf{表示动作的阶段}(正在进行、刚要开始、刚刚结束)。

在「ところ」后接「に」,表示\textbf{在前句动作的某个阶段上,发生了后句的动作。}

% 语法示例
\begin{example}
    \begin{itemize}
        \item 私が先生と\textbf{相談しているところに}、友達が訪れてきた。
        
        (我\textbf{正在}和老师商量\textbf{的时候},朋友来拜访了。)

        \item 私が\textbf{出かけるところに}、王さんが来ました。
        
        (我\textbf{正要}出门\textbf{的时候},王先生来了。)

        \item 夕食が\textbf{終わったところに}、客が来ました。
        
        (晚饭\textbf{刚}吃完\textbf{的时候},客人来了。)
    \end{itemize}
\end{example}

\separatorline

\section{词语与用法}
\subsection{「それから / それに」表并列}

% 语法结构
\begin{codeblock}
甲と乙と丙、それから丁

甲に乙に丙,それに丁
\end{codeblock}

% 语法解释
\textbf{「と」和「に」都是表示并列关系的。}「と」单纯地表示事物的并列,「に」则带有累加意义。

% 语法示例
\begin{example}
    \begin{itemize}
        \item テレビ\textbf{と}アイロン\textbf{と}扇風機、\textbf{それから}クーラー。
        \item 東京\textbf{に}名古屋\textbf{に}大阪、\textbf{それに}京都。
    \end{itemize}
\end{example}

\separatorline

\subsection{「なんか」表举例、列举}

% 语法结构
\begin{codeblock}
・・・なんか~
\end{codeblock}

% 语法解释
\textbf{「なんか」是「など」的口语形式},表示列举、举例。\textbf{相当于“・・・之类的~”。}

有时也能用于表达自谦,「私なんか~」表示“像我这种微不足道的人~”。

% 语法示例
\begin{example}
    \begin{itemize}
        \item \textbf{旅行なんか}、めったにできない。
        
        (\textbf{像旅行之类的事},很少能做。)

        \item \textbf{私なんか}、まだまだです。
        
        (像我这种人,还差得远呢。)
    \end{itemize}
\end{example}

\separatorline

\subsection{日常会话中的一些说法}

\subsubsection{・・・てちょうだい}
    
是「・・・てください」的口语形式,\textbf{表示请求。主要是女性使用。}

\begin{example}
    \begin{itemize}
        \item ちょっと\textbf{待ってちょうだい}。(请稍等一下。)
    \end{itemize}
\end{example}

\subsubsection{・・・かしら / かな}
    
表示「でしょうか」,\textbf{表示怀疑},有时也用与委婉地向他人询问。\textbf{「かしら」是女性用语,「かな」则男女皆可使用。}

\begin{example}
    \begin{itemize}
    \item 明日は雨が\textbf{降るかしら}。(明天会下雨吗?)
    
    \item 彼は\textbf{来るかな}。(他会来吗?)
    \end{itemize}
\end{example}

\subsubsection{・・・(ん)だなあ}

\textbf{「な」表示感叹,有时也用于自问自答。}「なあ」是其发音延长而成。

\begin{callout}
    这个语法是「のだ」+「なあ」的组合,「のだ」表陈述语气、强调,「なあ」用于感叹或自问自答。口语中,「のだ」常缩略为「んだ」。
    
    \vspace{10pt}

    其实,不使用「のだ」,「なあ」单独使用时,也能表达类似的感叹语气。\textbf{「なあ」单独使用时,前面通常接普通体。}
\end{callout}

\begin{example}
    \begin{itemize}
        \item もう夏休みが\textbf{終わるんだなあ}。(暑假就要结束了啊。)

        \item いい\textbf{天気だなあ}。(真是好天气啊。)
        
        \item 中国は人口が\textbf{多いなあ}。(中国人口真多啊。)
    \end{itemize}
\end{example}

\subsubsection{・・・さ}

用在句末,\textbf{表示说话者认为他的判断是客观的。男性用语。}

\begin{example}
    \begin{itemize}
        \item 彼は来ない\textbf{さ}。(他不会来的。)

        \item 今日は暑い\textbf{さ}。(今天很热的。)
        
        \item そんなこと、知ってる\textbf{さ}。(那种事,我知道的。)
    \end{itemize}
\end{example}