\chapter{標準日本語 第7課}

摘要:\textbf{無し}

\separatorline

\section{文法}

\subsection{「というふうに」表例示}

% 语法结构
\begin{codeblock}
・・・というふうに ~
\end{codeblock}

% 语法解释
表示例示,相当于\textbf{“像・・・那样”}。助词「と」提示内容。

% 语法示例
\begin{example}
    \begin{itemize}
        \item 今日は赤い帽子、明日は黄色い帽子\textbf{というふうに}、毎日違う色の帽子をかぶります。
        
        (今天戴红色的帽子,明天戴黄色的帽子,\textbf{像这样}每天戴不同颜色的帽子。)

        \item さくら、チューリップ\textbf{というふうに}、春にたくさんの花が咲きます。
    \end{itemize}
\end{example}

\separatorline
\subsection{「ことに」表感想}

% 语法结构
\begin{codeblock}
・・・ことに、 ~
\end{codeblock}

% 语法解释
在「ことに」的前面接表示情感的词,表示说话人对后句某事的感想、评价。相当于\textbf{“令人・・・的是~”}。

根据前接成分的不同,有三种形式:

\begin{tabularx}{\linewidth}{|X|X|}
\hline
前续成分 & 接续规则 \\
\hline
动词 & \textbf{た形}   + ことに \\
\hline
一类形容词 & 普通体  + ことに \\
\hline
形容动词 & \textbf{词干 + な}  + ことに \\
\hline
\end{tabularx}


% 语法示例
\begin{example}
    \begin{itemize}
        \item \textbf{驚いたことに}、あの二人は兄弟だったのです。
        
        (令人吃惊的是,那两个人竟然是兄弟。)

        \item \textbf{嬉しいことに}、試験に合格しました。
        
        (令人高兴的是,考试合格了。)

        \item \textbf{心配なことに}、妹はまだ帰って来ない。
    \end{itemize}
\end{example}

\separatorline

\subsection{「ではないでしょうか」表反问}

% 语法结构
\begin{codeblock}
・・・ではないでしょうか
\end{codeblock}

% 语法解释
向对方询问判断,说话人虽不能明确判断,但大致认为就是这样。是一种\textbf{反问的表达方式}。相当于\textbf{“不是・・・吗?”}“难道不是・・・吗?”。

\begin{callout}
    「ではないでしょうか」看似是表否定,但拆开来却是「ではない」(不是)+「でしょうか」(吗),合起来表示“不是・・・吗?”的意思。
\end{callout}

「ではないでしょうか」前面接续规则如下:

\begin{tabularx}{\linewidth}{|X|X|}
\hline
前续成分 & 接续规则 \\
\hline
动词 & 普通形 + \textbf{の} + ではないでしょうか \\
\hline
一类形容词 & 普通形 + \textbf{の} + ではないでしょうか \\
\hline
形容动词 & \textbf{词干 + な / だった} + ではないでしょうか \\
\hline
名词 & \textbf{词干 + だ / だった} + ではないでしょうか \\
\hline
\end{tabularx}

% 语法示例
\begin{example}
    \begin{itemize}
        \item 田中さんは\textbf{買ったのではないでしょうか}。
        
        (田中先生不是买了吗?)
        
        \item あの女性は、\textbf{寂しいの}ではないでしょうか。
        \item この場所のほうが、\textbf{静かなの}ではないでしょうか。
        \item あれは、あなたの\textbf{写真だった}ではないでしょうか。
    \end{itemize}
\end{example}

在形容动词和名词后面也可以直接接「ではないでしょうか」。

\begin{example}
    \begin{itemize}
        \item 彼の言葉使いは、\textbf{乱暴ではないでしょうか}。
        \item 彼は\textbf{犯人ではないでしょうか}。
    \end{itemize}
\end{example}

\separatorline

\subsection{「というものは」表讲述事物性质}

% 语法结构
\begin{codeblock}
・・・というものは、 ~
\end{codeblock}

% 语法解释
\textbf{表示・・・的事物具有~的性质},相当于“所谓・・・的东西是~”。

% 语法示例
\begin{example}
    \begin{itemize}
        \item \textbf{迷信というものは}、こうして生まれるのかもしれない。
        
        (\textbf{所谓迷信},或许就是这样产生的。)

        \item \textbf{新聞というものは}、新しい情報が書かれていなければ意味がない。
        \item \textbf{薬というものは}、間違って飲むと大変なことになる。
    \end{itemize}
\end{example}

\separatorline

\subsection{「がる」表第三人称情感}

% 语法结构
\begin{codeblock}
一类形容词\textbf{词干} + がる

形容动词\textbf{词干} + がる
\end{codeblock}

% 语法解释
日语中,表示情感的形容词和形容动词是表示说话人自己的感受的。\textbf{如果要表达第三人称的情感,则需要使用「がる」形式。}

\begin{callout}
    前一课所学的「たい」也是类似的用法,表示第三人称的愿望时,也需要使用「たがる」形式。
\end{callout}

% 语法示例
\begin{example}
    \begin{itemize}
        \item 純子さんは、テニスの試合に負けて\textbf{悔しがっています。}
        
        (纯子因为网球比赛输了而感到懊恼。)

        \item 佐藤さんは、病院に行くのを\textbf{嫌がっています。}
    \end{itemize}
\end{example}

\textbf{「・・・がる」整体可以看做一类动词},可以进行各种动词的变形。

\begin{example}
    \begin{itemize}
        \item そんなに\textbf{寂しがらなくて}もいい。
        
        (你不必那么觉得寂寞。)

        \item そんなに\textbf{恥ずかしがらないで}ください。
    \end{itemize}
\end{example}

\separatorline

\subsection{「ほど」表程度}

% 语法结构
\begin{codeblock}
・・・ほど ~
\end{codeblock}

% 语法解释
「・・・ほど」表示程度,相当于\textbf{“・・・那种程度”},一般表示程度很高(或很低,总体\textbf{强调“很”})。

\textbf{整体是副词},可以修饰动词、形容词、形容动词。

\textbf{「ほど」前接动词基本形、「ない」形。}

% 语法示例
\begin{example}
    \begin{itemize}
        \item 涙が\textbf{出るほど}痛かった。
        
        (疼得流眼泪。)

        \item 彼は、立っていることが\textbf{できないほど}お腹が空いている。
        \item \textbf{驚くほど}大きな地震だった。
    \end{itemize}
\end{example}

\separatorline

\newpage

\subsection{「くらいなら」表程度}

% 语法结构
\begin{codeblock}
・・・くらいなら、 ~
\end{codeblock}

% 语法解释
在「くらいなら」提出程度很高的情形,后面表示与其比较的其他情况。此语法代入句中更好理解。

% 语法示例
\begin{example}
    \begin{itemize}
        \item 先生がわからない\textbf{くらいなら}、私にわかるはずがない。
        
        (如果老师都不懂的话,我是不可能懂的。)

        \item いつも元気な山田さんも風邪をひく\textbf{くらいなら}、他の人は肺炎になってしまうだろう。
    \end{itemize}
\end{example}

\separatorline

\section{词语与用法}
\subsection{「によって」表原因}

% 语法结构
\begin{codeblock}
・・・によって、 ~
\end{codeblock}

% 语法解释
表示\textbf{原因},多用于\textbf{书面语}。「に」前面接名词或名词句。

% 语法示例
\begin{example}
    \begin{itemize}
        \item 大雨\textbf{によって}山が崩れた。
        
        (\textbf{由于}大雨山崩塌了。)

        \item いろいろな努力\textbf{によって}、工業が発達した。
    \end{itemize}
\end{example}

\separatorline

\subsection{「でも何でもない」表完全否定}

% 语法结构
\begin{codeblock}
[・・・ 形容动词\textbf{で}] も何\textbf{で}もない

[・・・   名词\textbf{で}] も何\textbf{で}もない

[・・・   动词て] も何\textbf{と}もない

[・・・一类形容词\textbf{く}] も何\textbf{と}もない
\end{codeblock}

% 语法解释
\textbf{是「全然・・・ではない」的强调说法。}根据前接成分的不同,有三种形式。


% 语法示例
\begin{example}
    \begin{itemize}
        \item 彼なんか、\textbf{友達で}もなん\textbf{で}もない。
        
        (他这种人,根本不是我朋友。)

        \item この道具を使ってみると、\textbf{便利で}も何\textbf{で}もない。
        \item そんなこと、\textbf{難しく}も何\textbf{と}もない。
        \item 彼はそんなこと、\textbf{気にして}も何\textbf{と}もない。
    \end{itemize}
\end{example}