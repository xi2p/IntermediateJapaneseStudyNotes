\chapter{標準日本語 第17課}

\separatorline

\section{文法}

\subsection{「心配はない」表不担心会发生某事}

% 语法结构
\begin{codeblock}
・・・心配はない

・・・心配はある
\end{codeblock}

% 语法解释
「心配はない」表示担心的某事不会发生,「心配はある」表示担心的某事可能会发生。

\textbf{「心配」是名词,其前连接规则和名词一致。}

% 语法示例
\begin{example}
    \begin{itemize}
        \item ここまで追ってくる\textbf{心配はない}。
        
        (\textbf{不用担心}会追到这里来。)

        \item こんなに雪が降ると、自動車が使えない\textbf{心配があります}。
        \item 事故の\textbf{心配はない}。
    \end{itemize}
\end{example}

除了「心配」以外,还有以下表达方式:

\begin{itemize}
    \item ・・・恐れがある / 恐れはない (担心的事情可能会发生/不会发生)
    \item ・・・可能性がある / 可能性はない  (有发生某事的可能/没有发生某事的可能)
    \item ・・・危険性がある / 危険性はない   (有发生某事的危险/没有发生某事的危险)
\end{itemize}

\begin{example}
    \begin{itemize}
        \item 電車の到着が遅れる\textbf{恐れがある}。
        
        (电车可能会晚点。)

        \item 君が落第する\textbf{可能性はない}から、安心しなさい。
    \end{itemize}
\end{example}

\separatorline
\subsection{「ずに済む」表不做某事,事情也能顺利解决}

% 语法结构
\begin{codeblock}
[・・・ず] に済む

[・・・ない] で済む
\end{codeblock}

% 语法解释
表示\textbf{不做某事,事情也能顺利解决},这对于说话人来说是很省事的。

% 语法示例
\begin{example}
    \begin{itemize}
        \item プレゼントとしてもらったので、\textbf{買わずに済んだ}。
        
        (因为作为礼物收到了,所以不用买了。)

        \item あなたが言ってくれたから、私は何も\textbf{言わずに済みました}。
        \item 買わ\textbf{ないで済んだ}。
        \item 何も言わ\textbf{ないで済んだ}。
    \end{itemize}
\end{example}

\separatorline

\subsection{「たものだ」表回忆过去状态}

% 语法结构
\begin{codeblock}
[・・・た] ものだ
\end{codeblock}

% 语法解释
表示\textbf{回忆起过去反复发生的事情或过去持续的状态}。此时,\textbf{「もの」前面必须接用言的「た」型}。

% 语法示例
\begin{example}
    \begin{itemize}
        \item 子供の頃、よくこの川で\textbf{泳いだものだ}。
        
        (小时候,经常在这条河里游泳。)

        \item 昔は、この辺りも家が少なくて、\textbf{静かだったものだ}。
    \end{itemize}
\end{example}

\separatorline

\subsection{「ものだ」表习惯或理所应当}

% 语法结构
\begin{codeblock}
・・・ものだ
\end{codeblock}

% 语法解释
表示\textbf{一般习惯这样做,或理所应当做的事情}。此时,\textbf{「もの」前面接用言的现在形式}。

% 语法示例
\begin{example}
    \begin{itemize}
        \item 人に会ったら、挨拶する\textbf{ものだ}。
        
        (人见面时,\textbf{应该}打招呼。)

        \item 年を取ると、体が弱くなる\textbf{ものだ}。
        
        (年纪大了,身体\textbf{就会}变弱。)
    \end{itemize}
\end{example}

\separatorline

\subsection{「たばかりに」表就是因为}

% 语法结构
\begin{codeblock}
[・・・た] ばかりに、 ~
\end{codeblock}

% 语法解释
表示因果关系,相当于\textbf{“就是因为・・・,结果~”}。多表示产生消极的影响。

% 语法示例
\begin{example}
    \begin{itemize}
        \item 私が旅行に誘っ\textbf{たばかりに}、彼は事故に遭った。
        
        (\textbf{就是因为}我邀请他去旅行,结果他遭遇了事故。)

        \item 正直に言っ\textbf{たばかりに}、怒られてしまった。
    \end{itemize}
\end{example}

\separatorline

\subsection{「わけにはいかない」表不能这样做}

% 语法结构
\begin{codeblock}
・・・わけにはいかない
\end{codeblock}

% 语法解释
表示\textbf{因为某些原因、情况,不能做・・・}。「わけ」是名词性质,其前面接动词普通体。

\begin{callout}
    此处原因多为情理、道义、社会规范、责任、承诺等方面的原因,而非客观上的不可能。
\end{callout}

% 语法示例
\begin{example}
    \begin{itemize}
        \item 病院にステレオを持っていく\textbf{わけにはいかない}。
        \item いくら痩せたくても、何も食べない\textbf{わけにはいかない}。
        
        (无论多想减肥,也\textbf{不能}什么都不吃。)
    \end{itemize}
\end{example}

\separatorline

\section{词语与用法}
\subsection{「のことだから」表因为・・・的特点}

% 语法解释
\begin{codeblock}
・・・のことだから、 ~
\end{codeblock}

用于名词之后。\textbf{「こと」表示名词的性质、特征}。在「だから」后断定或推测由其特征而引出的结果。

% 语法示例
\begin{example}
    \begin{itemize}
        \item 彼\textbf{のことだから}、心配しなくてもきっと試験に合格しますよ。
        
        (因为是他,所以不用担心,一定会通过考试的。)
    \end{itemize}
\end{example}

\separatorline

\subsection{「は間違いない」表断定}

\begin{codeblock}
・・・は間違いない
\end{codeblock}

% 语法解释
类似于「には違いない」,表示\textbf{“一定,没错”}。表示说话人确信是这么回事。

% 语法示例
\begin{example}
    \begin{itemize}
        \item 彼はよく勉強したから、試験に合格するの\textbf{は間違いない}。
        
        (他好好学习了,考试\textbf{一定}会通过的。)

        \item 将来,中国の工場で、もっと機械化が進むの\textbf{は間違いない}。
    \end{itemize}
\end{example}