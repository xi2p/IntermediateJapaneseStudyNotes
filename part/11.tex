\chapter{標準日本語 第11課}

摘要:\textbf{無し}

\separatorline

\section{文法}

\subsection{「どこへ・・・のにも」表无论・・・都・・・}

表示去\textbf{任何}场所,\textbf{都}如此。用其他疑问词可以构成同类的表达。

% 语法结构
\begin{codeblock}
疑问词 + 助词 + ・・・の + に + も、 ・・・
\end{codeblock}

% 语法解释
这里助词「の」是将动词名词化,助词「に」提示“在...情况下”,助词「も」表示“无论...都...”。

\textbf{此处助词「の」可以省略}。

% 语法示例
\begin{example}
    \begin{itemize}
        \item 兄は\textbf{どこへ}行く\textbf{のにも}、万年筆を持って行く。
        
        (哥哥\textbf{无论}去哪里,\textbf{都}带着钢笔。)

        \item \textbf{どこへ}行く\textbf{にも}、必ずこの駅を利用する。
        \item \textbf{だれと}話す\textbf{のにも}、丁寧に話さなければなりません。
        \item \textbf{何を}買う\textbf{のにも}、お金は必要だ。
    \end{itemize}
\end{example}

\separatorline

\subsection{「しかない」表只能}

% 语法结构
\begin{codeblock}
・・・しかない
\end{codeblock}

% 语法解释
表示\textbf{除了・・・以外没有别的选择},含有“只能・・・”的意思。通常用于否定句中。

「しかない」前面接\textbf{动词基本形},或接\textbf{表示手段、方法的名词}。

% 语法示例
\begin{example}
    \begin{itemize}
        \item 電話がない時代は、手紙を\textbf{書くしかなかった}。
        \item 誰も来ないので、\textbf{帰るしかなかった}。
        \item 東京へ行くには、この\textbf{電車しかない}。
    \end{itemize}
\end{example}

\separatorline

\subsection{「たとしても」表即使也}

% 语法结构
\begin{codeblock}
[・・・た] としても、 ~
\end{codeblock}

% 语法解释
表示\textbf{“即使・・・也~”},用于假设某种情况成立时,后项仍然如此。\textbf{前接各种词的「た形」}。

\begin{callout}
    这个语法的本质是前面学过的「とする」表假设。

    在使用「た形」这个点上,和「たら」有类似之处,都是是假设事情已发生,再讨论后续情况。
\end{callout}

% 语法示例
\begin{example}
    \begin{itemize}
        \item 私が\textbf{教えなかったとしても}、誰かが教えたでしょう。
        \item もし明日雨が\textbf{降ったとしても}、運動会は行われるでしょう。
        \item 彼の説明が\textbf{複雑だったとしても}、メモを取れば理解できるはずだ。
        \item 明日が\textbf{休日だったとしても}、私は出勤しなければならない。
    \end{itemize}
\end{example}

\separatorline

\subsection{「ものだ」表感叹}

% 语法结构
\begin{codeblock}
・・・ものだ
\end{codeblock}

% 语法解释
表示\textbf{感叹}的心情,\textbf{「もの」前接用言连接名词的形式。}

% 语法示例
\begin{example}
    \begin{itemize}
        \item 昔と比べると、最近はずいぶん便利に\textbf{なったものだ}。
        \item この試験で60点取れば、\textbf{立派なものだ}。
    \end{itemize}
\end{example}

注意到「もの」也能表示“东西,物品”的意思,所以「ものだ」不一定表示感叹。

\begin{example}
    \begin{itemize}
        \item この写真は、去年撮ったものだ。
    \end{itemize}
\end{example}

\separatorline

\newpage

\subsection{「ずにはいられない」表忍不住}

% 语法结构
\begin{codeblock}
[・・・ず] に(は)いられない

・・・ないではいられない
\end{codeblock}

% 语法解释
表示\textbf{忍不住做某事、无论如何都想这么做、不能不做某事}。

把动词「ない」形中的「ない」变成「ず」,即构成\textbf{「ず」形}。两者都表示否定。但「ず」形只在固定语法表达中使用,且较为书面语。

「いられない」是「いる」的可能态的否定形,表示“不能”。结合前面的「ず」形,表示“不能不・・・”。

% 语法示例
\begin{example}
    \begin{itemize}
        \item あの子がかわいそうで、\textbf{泣かずにはいられない}。
        \item 面白くて、最後まで\textbf{読まずにいられない}。
    \end{itemize}
\end{example}

「ないではいられない」也表达同样的意思,也是书面语的表达方式。

\begin{example}
    \begin{itemize}
        \item その意見には、\textbf{反対しないではいられない}。
    \end{itemize}
\end{example}

\separatorline

\section{词语与用法}
\subsection{「になって」表时间到达}

% 语法结构
\begin{codeblock}
[・・・时间] になって
\end{codeblock}

% 语法解释
表示\textbf{“到了・・・”}。

% 语法示例
\begin{example}
    \begin{itemize}
        \item \textbf{春になって}、旅行に行く人が増えました。
    \end{itemize}
\end{example}

\separatorline

\subsection{「もしないうちに」表不到・・・就・・・}

% 语法结构
\begin{codeblock}
[・・・时间] もしないうちに、 ~
\end{codeblock}

% 语法解释
表示\textbf{“还没到・・・的时候,就已经・・・了”}。

\begin{callout}
    这个语法是「も」 + 「しない」 + 「うちに」的组合。
\end{callout}

% 语法示例
\begin{example}
    \begin{itemize}
        \item 王さんに手紙を出したら、\textbf{3日もしないうちに返事が来た}。
    \end{itemize}
\end{example}

\separatorline

\subsection{「わずか・・・足らず」}

% 语法结构
\begin{codeblock}
わずか [・・・时间] 足らず~
\end{codeblock}

% 语法解释
\textbf{「わずか」表示“仅仅,只有”,「足らず」表示“不足,不到”}。

「わずか」主要用于书面语。

% 语法示例
\begin{example}
    \begin{itemize}
        \item 東京から京都まで、\textbf{3時間足らず}行けるようになりました。
        \item コンピュータなら、この計算が、\textbf{わずか}10秒\textbf{足らず}でできます。
    \end{itemize}
\end{example}

\separatorline

\subsection{「たつ」表出发}
% 语法解释
表示\textbf{出发},出发地点用「を」或者「から」提示。

% 语法示例
\begin{example}
    \begin{itemize}
        \item 明日朝7時に東京を\textbf{たつ}予定です。
    \end{itemize}
\end{example}

\separatorline