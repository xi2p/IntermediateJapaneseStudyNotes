\chapter{標準日本語 第8課}

摘要:\textbf{無し}

\separatorline

\section{文法}

\subsection{「ばかり」表不仅,而且}

% 语法结构
\begin{codeblock}
・・・ばかりか、 ~も・・・
\end{codeblock}

% 语法解释
表示不仅・・・,而且・・・。

「ばかり」前面可以接\textbf{动词/一类形容词普通体},或接\textbf{形容动词+な}。


% 语法示例
\begin{example}
    \begin{itemize}
        \item あなたが困る\textbf{ばかりか}、みんな\textbf{も}迷惑する。
        \item 彼女は美しい\textbf{ばかりか}、性格\textbf{も}優しい。
        \item 田中さんは陽気\textbf{なばかりか}、他人に\textbf{も}親切だ。
    \end{itemize}
\end{example}

\separatorline
\subsection{「とする」表假设}

% 语法结构
\begin{codeblock}
・・・とする

・・・としたら/とすれば/とすると/としても、 ~
\end{codeblock}

% 语法解释
表示假设,相当于中文的\textbf{“假设・・・”}。「と」前接普通体。

% 语法示例
\begin{example}
    \begin{itemize}
        \item 7642円もらえる\textbf{とする}。(假设能得到7642日元。)
    \end{itemize}
\end{example}

通过变成「としたら」「とすれば」「とすると」「としても」等形式,可以后接假设产生的结果。

\begin{example}
    \begin{itemize}
        \item 毎日5キロ歩く\textbf{としたら}、一年で1825キロ歩くことになる。
        \item 次の試験も難しい\textbf{とすれば}、合格は難しいだろう。
        \item もしこの時計は正確だ\textbf{としたら}、今は3時15分だ。
        \item 明日は雨だ\textbf{としても}、試合は行われるだろう。(\textbf{即使}明天下雨,比赛也会进行。)
    \end{itemize}
\end{example}


\separatorline

\subsection{「と言えば かもしれないけど」表这么说也可以,但是}

% 语法结构
\begin{codeblock}
・・・といえば、 ・・・かもしれないけど、 ~
\end{codeblock}

% 语法解释
表示\textbf{“说是・・・也可以,但是~”}。句中的两处「・・・」应该是相同内容。

「・・・」处接续规则同「かもしれない」。

% 语法示例
\begin{example}
    \begin{itemize}
        \item あの人は美しい\textbf{と言えば}美しい\textbf{かもしれないけど}、あまり目立たない。
        \item ここは静か\textbf{と言えば}静か\textbf{かもしれないけど}、不便だ。
    \end{itemize}
\end{example}

第一处「・・・」也可以接普通体。

\begin{example}
    \begin{itemize}
        \item ここは静か\textbf{だと言えば}静か\textbf{かもしれないけど}、不便だ。
        \item あの人は美しい\textbf{と言えば}美しい\textbf{けど}、あまり目立たない。
    \end{itemize}
\end{example}

\begin{callout}
    这个语法实际是「・・・と言えば」和「・・・かもしれない」的结合体。

    \vspace{10pt}

    在接续规则上,可以全部使用「かもしれない」的规则(如第一处例子所示),也可以前面用「と言えば」的规则,后面用「かもしれない」的规则(如第二处例子所示)。
\end{callout}

\separatorline

\subsection{「みたい」表推测、比喻、示例}

% 语法结构
\begin{codeblock}
・・・みたいだ
\end{codeblock}

% 语法解释
\textbf{类似于「ようだ」},但语气较\textbf{口语化}。女性口语中,有时会省略「だ」。

「みたい」前面接\textbf{动词/形容词的普通体},或接\textbf{形容动词词干},或接\textbf{名词}。

可以「みたいな」修饰名词。可以用「みたいに」作副词。

% 语法示例


\begin{example}
    可以表示推测:
    \begin{itemize}
        \item 雨がやんだ\textbf{みたいだ}。
        \item 君のほうが料理が上手\textbf{みたいだ}。
    \end{itemize}
    
    \vspace{10pt}

    可以表示比喻:

    \begin{itemize}
        \item あの人はとてもきれいで、花\textbf{みたい}。
        \item 彼の字は、まるで小学生が書いた\textbf{みたいに}、下手だ。
    \end{itemize}

    \vspace{10pt}

    可以表示示例:

    \begin{itemize}
        \item 君\textbf{みたいな}人は、あまりいない。
        \item ケーキ\textbf{みたいな}甘いものは、あまり食べない。
    \end{itemize}
\end{example}

\separatorline

\section{词语与用法}
\subsection{「てくれない」表请求}

% 语法结构
\begin{codeblock}
・・・てくれない
\end{codeblock}

% 语法解释
\textbf{是「てくださいませんか」的简体},对亲近的人表示委托或请求。

% 语法示例
\begin{example}
    \begin{itemize}
        \item ちょっと手伝っ\textbf{てくれない}。(母亲对孩子说)
        \item 田中君、これコピーし\textbf{てくれない}。(上司对下属说)
    \end{itemize}
\end{example}

\separatorline

\subsection{「につき」表每}

% 语法结构
\begin{codeblock}
・・・につき、 ~
\end{codeblock}

% 语法解释
是「について」的书面语,表示\textbf{“每・・・”}。

% 语法示例
\begin{example}
    \begin{itemize}
        \item 1個\textbf{につき}、運び賃を10円もらえる。
        (\textbf{每}个获得搬运费10日元。)
    \end{itemize}
\end{example}

\separatorline

\subsection{「てごらん」表试试看}

% 语法结构
\begin{codeblock}
・・・てごらん
\end{codeblock}

% 语法解释
类似于「てみなさい」,表示\textbf{“试试看・・・”}。只有关系亲近的长辈对晚辈使用。

% 语法示例
\begin{example}
    \begin{itemize}
        \item これ食べ\textbf{てごらん}。おいしいわよ。(母亲对孩子说)
    \end{itemize}
\end{example}

\separatorline