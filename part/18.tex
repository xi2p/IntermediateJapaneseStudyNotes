\chapter{標準日本語 第18課}

\separatorline

\section{文法}

\subsection{「にともなって / とともに」表随着}

% 语法结构
\begin{codeblock}
・・・にともなって、 ~

・・・とともに、 ~
\end{codeblock}

% 语法解释
表示“随着・・・”。

「に」和「と」的前面接动词的基本形,或表示变化的名词。

% 语法示例
\begin{example}
    \begin{itemize}
        \item 自動車の数が増えるにともなって、交通事故も多くなった。
        
        (随着汽车数量的增加,交通事故也变多了。)

        \item 外国での生活に慣れるとともに、よく眠れるようになりました。
        \item 気温の上昇にともなって、クーラーもたくさん売れるようになります。
        \item 産業の発達とともに、環境の汚染が問題になってきました。
    \end{itemize}
\end{example}

\begin{callout}
    此前学过的「にしたがって」也能表示“随着・・・”,但它前面应接动词基本形。
\end{callout}

\separatorline

\subsection{「以上」表前提}

% 语法结构
\begin{codeblock}
・・・以上、 ~
\end{codeblock}

% 语法解释
以「以上」前的事情为前提,发表说话人的主观判断。相当于“既然・・・”。

「以上」前面接「动词/一类形容词 普通体」,或接「名词/形容动词 + である」。

% 语法示例
\begin{example}
    \begin{itemize}
        \item 一度約束した以上、自分で責任を持たなければなりません。
        
        (既然已经约定了,就必须自己负责。)

        \item お金がない以上、我慢するしかない。
        \item この大学の学生である以上、校則を守らなければならない。
        \item この本が必要である以上、他人に貸すわけにはいかない。
    \end{itemize}
\end{example}

\separatorline

\subsection{「ざるを得ない」表不能不}

% 语法结构
\begin{codeblock}
[・・・ざる] を得ない
\end{codeblock}

% 语法解释
表示“不能不”,类似于之前学的「~ないわけにはいかない」,但「ざるを得ない」更正式,多用于书面语。

类似于「ず」形,把「ない」形的「ない」用「ざる」替换,就构成了「ざる」形。
特别地,「する」要变化成「せざる」。

% 语法示例
\begin{example}
    \begin{itemize}
        \item 誰もいないようなので、帰らざるを得ない。
        
        (好像没人了,只好回去了。)

        \item 大人しくせざるを得ない。
        \item あなたの言葉なら、信用せざるを得ない。
        
        = あなたの言葉なら、信用しないわけにはいかない。
    \end{itemize}
\end{example}

\separatorline

\subsection{「だけでは済まない」表只做・・・还不能解决问题}

% 语法结构
\begin{codeblock}
・・・だけでは済まない

\end{codeblock}

% 语法解释
表示“只做・・・还不能解决问题”。

类似地,「だけで済む」表示“只做・・・就能解决问题”。

「だけ」前面接动词普通体,或接表示动作的名词。

% 语法示例
\begin{example}
    \begin{itemize}
        \item 新しい部品を取り替えるだけでは済まない。 
        
        (只更换新零件还不能解决问题。)

        \item 電話だけでは済まない大事な要件です。
        \item 部品を取り替えなくても、修理するだけで済む。
        \item 会いに行かなくても、電話だけで済みます。
    \end{itemize}
\end{example}

\begin{callout}
    语法中的「では」可以拆分成「で」+「は」。其中「で」表示手段、方法,「は」表示强调。

    \vspace{10pt}

    「だけでは済まない」直译为“只用・・・的方法/手段,是不能解决问题的”。
\end{callout}

\separatorline

\subsection{「(とかく)がちだ」表倾向}

% 语法结构
\begin{codeblock}
(とかく)・・・がちだ
\end{codeblock}

% 语法解释
表示某事经常发生或总是形成某种状态。主要是指不好的事或状态。常与「とかく」一起使用。

\begin{callout}
    「がち」源于「勝ち」,表示倾向。倾向于某种状态,即某事经常发生或总是形成某种状态。

    \vspace{10pt}

    类似的表达有此前学过的「やすい」。但「やすい」侧重于“容易做某事”,而「がち」侧重于“经常发生某事”,且多指不好的事或状态。
\end{callout}

% 语法示例
\begin{example}
    \begin{itemize}
        \item 幸福な時は、とかく他人のことを忘れがちだ。 (幸福的时候,人们往往容易忘记他人。)

        \item 地震の時は、とかく慌てがちだ。
        \item あの家は留守がちだ。
        \item 私の母は病気がちだ。
    \end{itemize}
\end{example}

\separatorline

\section{词语与用法}
\subsection{「における」表在}

% 语法结构
\begin{codeblock}
・・・における~
\end{codeblock}

% 语法解释
此前学过「において」表示“在”。当后续要修饰名词时,使用「における」,表示场所或时间。

% 语法示例
\begin{example}
    \begin{itemize}
        \item 1987年における自動車の生産台数は、500万台を超えた。
        
        (1987年汽车的生产台数超过了500万辆。)

        \item 北京における彼のスピーチは、たいへんな評判になった。
    \end{itemize}
\end{example}

\separatorline

\subsection{「当たり」表每}

% 语法结构
\begin{codeblock}
・・・当たり
\end{codeblock}

% 语法解释
表示“每・・・”。常与数量词连用,表示单位。

% 语法示例
\begin{example}
    \begin{itemize}
        \item 米1キログラム当たりの値段は、この3ヶ月で2倍になった。
        \item 食事1回当たり、10\%の税金がかかります。
    \end{itemize}
\end{example}

\begin{callout}
    此前学过「につき」也表示“每”:「入場料は一人につき500円です。」
\end{callout}