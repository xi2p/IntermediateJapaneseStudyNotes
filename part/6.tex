\chapter{標準日本語 第6課}

摘要:\textbf{無し}

\separatorline

\section{文法}

\subsection{「通り」表正如}

% 语法结构
\begin{codeblock}
・・・通り、 ~
\end{codeblock}

% 语法解释
表示\textbf{“正如・・・一样”},“依照・・・”。其中,「通り」假名为「とおり」。

「通り」可以看做名词性质,前面可以接\textbf{动词的普通形};名词和\textbf{一些副词}后面接「\textbf{の}通り」。

「・・・通り」\textbf{整个短语可以表现出副词性质},可以直接修饰句子、动词。

% 语法示例
\begin{example}
    \begin{itemize}
        \item 彼女は写真で\textbf{見た通り}、たいへん美しい女性です。
        \item 地図に\textbf{書いてある通り}、右の方に高い山が見えました。
        \item \textbf{説明の通り}、機械を操作してください。
        \item 道路は\textbf{いつもの通り}混雑しています。
    \end{itemize}
\end{example}

「通り」可以后接「に」,也可以用于修饰动词;「通り」后接「だ」,可以用作谓语。

\begin{example}
    \begin{itemize}
        \item この本に\textbf{書いてある通りに}、部品を組み立ててください。
        \item 先生が\textbf{発音する通りに}、声を出してください。
        \item 機械の構造は本に\textbf{書いてある通りでした}。
    \end{itemize}
\end{example}

\separatorline

\subsection{「てならない」表无法抑制}

% 语法结构
\begin{codeblock}
[・・・て] ならない
\end{codeblock}

% 语法解释
表示某种感觉或状态\textbf{强烈到无法抑制}的程度,相当于\textbf{“・・・得受不了” “・・・得不得了”}。

接续要求类似于「てたまらない」,但\textbf{「てならない」还可以接动词的「て」形}。

\newpage

接表达人的生理、情感的动词、一类形容词的「て」形:

% 语法示例
\begin{example}
    \begin{itemize}
        \item 涙が\textbf{出てなりませんでした}。
        
        \item 私は、失敗するような\textbf{気がしてならない}。
        
        (我\textbf{总}觉得会失败)

        \item あの人がいなくなってから、\textbf{寂しくてならない}。
    \end{itemize}
\end{example}

接表达情感的二类形容词的「て」形:

\begin{example}
    \begin{itemize}
        \item 式の試験が\textbf{不安でならない}。
        \item 新しい製品の評判が\textbf{気がかりでならない}。
    \end{itemize}
\end{example}

\separatorline

\subsection{「た途端に」表动作刚完成就・・・}

% 语法结构
\begin{codeblock}
[・・・动词た] 途端(に)、 ~
\end{codeblock}

% 语法解释
\textbf{表示・・・动作刚完成,紧接着就发生了~。}相当于中文的“一・・・就・・・”。

「た途端に」的假名为「たとたんに」。\textbf{有时也可以省略「に」。}

% 语法示例
\begin{example}
    \begin{itemize}
        \item 車から降り\textbf{た途端に}、転んでしまった。
        \item 彼女は大学を卒業し\textbf{た途端に}、結婚しました。
        \item 窓を開け\textbf{た途端}、冷たい風が部屋の中に入ってきました。
    \end{itemize}
\end{example}

\begin{callout}
    这个语法里,前项动作是已经完成的,\textbf{并且说话人确定其已经完成了。}
\end{callout}

\separatorline

\subsection{「てばかりいる」表一直做某事}

% 语法结构
\begin{codeblock}
・・・てばかりいる
\end{codeblock}

% 语法解释
表示一直做某事。初级已经学过此语法,此处再强调一下\textbf{要使用「ている」。}

% 语法示例
\begin{example}
    \begin{itemize}
        \item 働い\textbf{てばかりいる}と、疲れてしまいますよ。
        \item \textbf{休んでばかりいないで}、少しは手伝ってください。
    \end{itemize}
\end{example}

\separatorline

\subsection{「たがっている」表第三人称想要}

% 语法结构
\begin{codeblock}
[・・・动词连用形] たがっている

[・・・动词连用形] たがる
\end{codeblock}

% 语法解释
表示\textbf{第三人称的愿望或欲望},相当于“想要・・・”。此前学过「たい」表示第一人称和第二人称的愿望。

「たがる」用于表示瞬间的愿望,「たがっている」用于表示持续的愿望。

\textbf{一般会使用「たがっている」的形式。}

% 语法示例
\begin{example}
    \begin{itemize}
        \item 王さんはジュースを\textbf{飲みたがっている。}
        \item 佐藤さんは旅行に\textbf{行きたがっている。}
    \end{itemize}
\end{example}

在使用「たい」时,用助词「が」表示动作的宾语,\textbf{而使用「たがる」时,宾语用助词「を」。}

\begin{example}
    \begin{itemize}
        \item わたしはジュース\textbf{が}飲みたいです。
        \item 王さんはジュース\textbf{を}飲みたがっている。
    \end{itemize}
\end{example}

\separatorline

\section{词语与用法}
\subsection{「ある」表某个}

% 语法结构
\begin{codeblock}
ある~
\end{codeblock}

% 语法解释
\textbf{表示“某个・・・”,“有一个・・・”。}

% 语法示例
\begin{example}
    \begin{itemize}
        \item \textbf{ある人}は日本人は小さいものが好きだと言いました。
        (\textbf{有个人}说日本人喜欢小东西。)

        \item \textbf{ある日}、彼は突然会社を辞めました。
        
        \item \textbf{ある時}、彼女は私に手紙を書きました。
        
        (\textbf{有一次},她给我写了封信。)

        \item \textbf{ある国}では8月に雪が降り、\textbf{ある国}では2月に雪が降る。
        
        (\textbf{有的国家}8月下雪,\textbf{有的国家}2月下雪。)

    \end{itemize}
\end{example}