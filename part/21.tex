\chapter{標準日本語 第21課}

\separatorline

\section{文法}

\subsection{「にすぎない」表只不过}

% 语法结构
\begin{codeblock}
・・・にすぎない
\end{codeblock}

% 语法解释
表示“\textbf{只不过}(是)・・・”,通常作为文章用语使用。口语中可用「だけだ」。

\textbf{「にすぎない」前面接动词普通体,或名词。}

% 语法示例
\begin{example}
    \begin{itemize}
        \item フランス語ができると言っても、大学で1年間勉強した\textbf{にすぎない}。
        \item 私はただ当然のことをした\textbf{にすぎない}。
        \item これは、私の個人的な意見\textbf{にすぎない}。
        \item 今年の収穫は去年の半分\textbf{にすぎない}。
    \end{itemize}
\end{example}

\separatorline
\subsection{「うえで」表关于}

% 语法结构
\begin{codeblock}
・・・うえで、 ~
\end{codeblock}

% 语法解释
表示“关于・・・”,“在・・・方面”,“在・・・上”,多用于书面语。

\textbf{「うえで」前接动词基本形(或た形)、名词+の。}

% 语法示例
\begin{example}
    \begin{itemize}
        \item 就職先を決める\textbf{うえで}、先生に色々相談に乗っていただいた。
        \item 運動することは、健康を保つ\textbf{うえで}、重要なことである。
        \item この論文は、書き方の\textbf{うえで}、問題がある。
        \item 英語とフランス語は、発音の\textbf{うえで}、大きな違いがある。
    \end{itemize}
\end{example}

\begin{callout}
    此前学过「うえに」,表示更进一层,“而且”,“不仅・・・还・・・”,与「うえで」不同。

    \begin{itemize}
        \item 彼は遅れた\textbf{うえに}、宿題も忘れた。
    \end{itemize}
\end{callout}

\separatorline

\subsection{「一方で」表另一方面}

% 语法结构
\begin{codeblock}
・・・一方で、 ~
\end{codeblock}

% 语法解释
「一方で」本身可以作为接续词,表示\textbf{“另一方面”}。也可以接在词的后面,表示“・・・一方面,另一方面・・・”,并列叙述两个对照性事物。

\textbf{「一方で」前接动词基本形。}

% 语法示例
\begin{example}
    \begin{itemize}
        \item 洋子さんは、勉強ができる\textbf{一方で}、スポーツも得意だ。
        \item 詳しく事故の様子を知りたいと思う\textbf{一方で}、知ることに不安を感じている。
        \item この町は夏は非常に蒸し暑い。\textbf{一方で}、冬はとても寒い。
    \end{itemize}
\end{example}

\separatorline

\subsection{「ことはない」表用不着}

% 语法结构
\begin{codeblock}
・・・ことはない
\end{codeblock}

% 语法解释
表示\textbf{“用不着・・・”},类似于「・・・なくてもよい」。

「こと」前接动词基本形。

% 语法示例
\begin{example}
    \begin{itemize}
        \item 失敗を恐れる\textbf{ことはない}。
        \item 小さな自信なのだから、慌てる\textbf{ことはない}。
    \end{itemize}
\end{example}


\begin{callout}
    要注意,「ことはない」本身也能单纯表示“不会有某事”的意思。

    \begin{itemize}
        \item こんなに晴れているし、今夜が雨が降ることはないでしょう。
    \end{itemize}
\end{callout}

\separatorline

\subsection{「まさか・・・ない」表没有意料到}

% 语法结构
\begin{codeblock}
まさか・・・ないだろう

まさか・・・とは思わなかった
\end{codeblock}

% 语法解释
表示\textbf{没有意料到}。

以「まさか・・・ないだろう」的形式,表示说话人没有预料到某事的发生。

以「まさか・・・とは思わなかった」的形式,表示说话人没有预料到某事的发生,\textbf{且此事已经发生}。“真没想到”

% 语法示例
\begin{example}
    \begin{itemize}
        \item \textbf{まさか}彼は知ら\textbf{ないでしょう}。
        \item \textbf{まさか}彼は本気じゃ\textbf{ないでしょう}。
        \item \textbf{まさか}、わたしがクラスで一番になる\textbf{とは思いません}。
        
        (真没想到我会成为班上第一名。)
    \end{itemize}
\end{example}

\begin{callout}
    「まさか・・・ないだろう」的语义存在一定争议,笔者也不能很好确定。

    例句第一句中的「まさか彼は知らないでしょう」在网络上存在两种理解:
    \begin{itemize}
        \item 说话人没有预料到“他不知道”这件事。
        \item 说话人预料到“他不知道”这件事。
    \end{itemize}

    但不管怎么样,这个表达核心意思都是“没有意料到”,发生的事情与自己预期不符。

\end{callout}
\separatorline

\section{词语与用法}
\subsection{「もの」表理由}

% 语法结构
\begin{codeblock}
・・・もの(ね)
\end{codeblock}

% 语法解释
「もの」是\textbf{表示理由}的终助词,常用于\textbf{口语}中,表示“因为・・・”。\textbf{主要是女性使用}。

「もん」是「もの」的音变,男女皆可使用。

% 语法示例
\begin{example}
    \begin{itemize}
        \item A: 「昼休みに公園を散歩したら、気持ちよかったです。」
        
        B: 「今日はいい天気\textbf{ですものね}。」

        \item A: 「今日は眠かったわ。」
        
        B: 「昨日夜遅くまで勉強した\textbf{もんね}。」

        \item A: 「最近疲れているみたいね。」
        
        B: 「仕事が忙しい\textbf{もの}。」

    \end{itemize}
\end{example}