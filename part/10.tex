\chapter{標準日本語 第10課}

摘要:\textbf{無し}

\separatorline

\section{文法}

\subsection{「のことだ」表下定义}

% 语法结构
\begin{codeblock}
[甲]とは    [乙]のことだ

\vspace{10pt}

[甲]というのは [乙]のことだ
\end{codeblock}

% 语法解释
表示给事物\textbf{下定义}。甲是要定义的事物的名词,乙是对甲的解释说明。

% 语法示例
\begin{example}
    \begin{itemize}
        \item 週刊誌\textbf{とは}、毎週1回出る雑誌\textbf{のことです}。
        
        (周刊杂志\textbf{是指}每周出版一次的杂志。)

        \item アクセサリー\textbf{とは}、身に付ける飾り\textbf{のことです}。
        \item 画家\textbf{というのは}、絵を描く仕事をする人\textbf{のことです}。
        \item 地下鉄\textbf{というのは}、地面の下を走っている鉄道\textbf{のことだ}。
    \end{itemize}
\end{example}

\separatorline

\subsection{「て形」表请求}

% 语法解释
动词 + 「てください」可以表达请求。在口语中,可以省略「ください」,只\textbf{用动词的て形来表达请求}。

这是一种熟不拘礼节的表达方式,多用于朋友或关系亲密的人之间。

% 语法示例
\begin{example}
    \begin{itemize}
        \item 私の家に遊びに\textbf{来て}。
        
        (请来我家玩。)

        \item 今朝の新聞を\textbf{読んで}。
    \end{itemize}
\end{example}


\separatorline

\subsection{「ですって / だって」表传闻}

% 语法结构
\begin{codeblock}
・・・ですって

・・・だって
\end{codeblock}

% 语法解释
\textbf{传达从他人处听到的话}。相当于“听说・・・” “据说・・・”的意思。

接名词时,用「名词 + だって」; 接动词/一类形容词时,用\textbf{「动词/一类形容词 + ん + だって」}。

% 语法示例
\begin{example}
    \begin{itemize}
        \item 明日は雪が降るかもしれ\textbf{ないんだって}。
        
        (\textbf{听说}明天可能会下雪。)

        \item 明日は\textbf{晴れですって}。
    \end{itemize}
\end{example}

此前所学的「そうだ / ということだ」也有传闻的意思,但「ですって / だって」更口语化,更常用于日常会话中。

\begin{example}
    \begin{itemize}
        \item 映画を\textbf{見るんですって}。
        
        映画を\textbf{見るそうです}。

        映画を\textbf{見るということです}。
    \end{itemize}
\end{example}

\begin{callout}
    \textbf{「って」是「と」的口语形式},提示内容。本语法的本质是用「・・・と言う」。「と」前应接普通体。

    \vspace{10pt}

    对于名词和形容动词,可以直接用「・・・だと言う」,因为「・・・だ」是它们的普通体。

    \vspace{10pt}

    对于动词和一类形容词,则需要用「・・・んだと言う」。这本质上是利用「の」,把动词/一类形容词变成名词性短语,从而与名词实现形式上的统一。
    由于「の」在口语中常读作「ん」,所以变成了「・・・んだと言う」。但即使不音变,也是可以的,即「・・・のだと言う」。

    \vspace{10pt}

    实际上,对于动词和一类形容词,也可以直接用普通体接「と」,即普通体接「って」。
\end{callout}

\begin{example}
    \begin{itemize}
        \item 彼は知って\textbf{いるって}
        
        \item 彼は\textbf{忙しいって}。
        
        \item 知り\textbf{たいのだって}。
    \end{itemize}
\end{example}