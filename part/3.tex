\chapter{標準日本語 第3課}

摘要:\textbf{無し}

\separatorline

\section{文法}

\subsection{「には」表目的}

\begin{codeblock}
[・・・动词基本形] には、~
\end{codeblock}

「には」前面是目的,后面是行为,表示\textbf{为了达到・・・目的而进行~行为}。

「には」前是动词基本形。

\begin{example}
    \begin{itemize}
        \item 医者に\textbf{なるには}、難し試験に合格することが必要です。
        \item 山田さんの学校へ\textbf{行くには}、電車に乗るのが一番いい方法だ。
    \end{itemize}
\end{example}

\separatorline

\subsection{动词的连用中止}

\begin{callout}
    动词连用形在初级阶段已经学习过,简单来说是「动词ます形」去掉「ます」后的形式。
\end{callout}

“连用中止”是指\textbf{使用连用形来中断句子}。此前学习过的「て」形也具有中止的作用。

被“连用中止”的句子间可以有多种关系,有:

\begin{center}
    \textbf{并列 ・ 对比 ・ 顺序 ・ 原因 ・ 状态}
\end{center}


\begin{example}
    \begin{itemize}
        \item 本を\textbf{読み}、音楽を聞きます。     (并列)
        \item 昼は\textbf{働き}、夜は勉強します。     (对比)
        \item 朝ご飯を\textbf{食べ}、学校へ行きます。   (顺序)
        \item 手が\textbf{震え}、うまく書けませんでした。 (原因)
        \item 力を\textbf{入れ}、体を動かしてください。  (状态)
    \end{itemize}
\end{example}

「て」形作中止作用,也能表达上述关系,但\textbf{连用中止更正式,更书面化},多用于书面语。

\separatorline

\subsection{「とよい」表建议}

\begin{codeblock}
[・・・动词基本形] とよい

[・・・动词基本形] といい
\end{codeblock}

表示\textbf{委婉地劝告做・・・}。做某事会得到好的结果。

「とよい」和「といい」意思相同,但「とよい」更正式,更书面化。

\begin{callout}
    此语法是由「と」和「よい / いい」构成的复合句型,直译为“如果做・・・的话,会好的”。
\end{callout}

\begin{example}
    \begin{itemize}
        \item 困ったことがあったら、先生に\textbf{相談するといい}。
        \item もっと本を読みたいのなら、あそこの図書館に\textbf{行くといい}ですよ。
    \end{itemize}
\end{example}

\separatorline

\subsection{「もう少しで・・・そうになった」表差点}

\begin{codeblock}
もう少しで [・・・动词连用形] そうになった

もう少しで [・・・动词基本形] ところだった
\end{codeblock}

表示\textbf{“差点就・・・了”}。表示某事差一点就要发生,但最终没有发生。

\begin{example}
    \begin{itemize}
        \item \textbf{もう少しで転びそうになった}。
        \item 事故に遭って、\textbf{もう少しで死にそうになった}。
    \end{itemize}
\end{example}

「そうになった」可以用「ところだった」替换,意思大体相同。

「少しで」可以替换为「ちっょと」。相比之下「ちっょと」更口语化,意思更广。

\textbf{注意:「そうになった」和「ところだった」前面接的动词形式不同。}

\begin{example}
    \begin{itemize}
        \item \textbf{もうちょっとで先生に怒られるところでした。}
        \item \textbf{もうちょっとでけがをするところでした。}
    \end{itemize}
\end{example}

\separatorline

\newpage

\subsection{「形容词 + め」表稍微・・・}

\begin{codeblock}
一类形容词词干 + め
\end{codeblock}

一类形容词词干 + め,表示\textbf{“稍微・・・一些”}。

\textbf{变形后的词是名词性质}。可以接续助词「に」构成副词短语。可以接续助词「の」修饰名词。

\begin{example}
    \begin{itemize}
        \item 子供の成長は早いので、\textbf{大きめの}洋服を買いましょう。
        \item 友達が大勢来るので、料理を\textbf{多めに}用意しました。
    \end{itemize}
\end{example}


\separatorline



\section{词语与用法}

\subsection{「除く」表除了}

表示\textbf{“除了・・・之外”。}

\begin{example}
    \begin{itemize}
        \item \textbf{日曜日を除く}毎日、彼はジョギングをしています。
        
        \item そして、用意した具を、\textbf{卵を除いて}順番に混ぜ合わせる。
        
        (这里的“卵を除いて”表示“除了鸡蛋之外”)
    \end{itemize}
\end{example}

