\chapter{標準日本語 第24課}

\separatorline

\section{文法}

\subsection{「ばかりに」表(修辞的)强调程度之甚}

% 语法结构
\begin{codeblock}
・・・ばかりに~
\end{codeblock}

% 语法解释
\textbf{(修辞地)强调某种程度之甚},是文学作品中常用的表达方式。\textbf{“仿佛・・・一般”}。整个短句作为副词使用。

\textbf{「ばかり」前面接动词基本形。}

% 语法示例
\begin{example}
    \begin{itemize}
        \item きらめく\textbf{ばかりに}白い砂浜が続いている。
        
        (闪闪发光般的白色沙滩绵延不绝。)

        \item 舌がとろける\textbf{ばかりに}美味しいケーキだった。
        
        (美味得仿佛舌头都要融化了一般的蛋糕。)

        \item 体が凍る\textbf{ばかりに}冷たい北風が吹いている。
    \end{itemize}
\end{example}

\separatorline

\subsection{「んばかりに」表仿佛现在就要・・・一般}

% 语法结构
\begin{codeblock}
[・・・ん] ばかりに~
\end{codeblock}

% 语法解释
表示「今にも・・・しそうになるほど」的意思,即\textbf{“仿佛现在就要・・・一般”、“眼看就要・・・”、“几乎快要・・・”}。整个短句作为副词使用。

「・・・ん」是古代文言的用法,表推量。\textbf{将「ない形」中的「ない」替换成「ん」即可得到「・・・ん」。}

% 语法示例
\begin{example}
    \begin{itemize}
        \item その男が今にも\textbf{倒れんばかりに}、ふらふら走っている。
        \item まるで窓ガラスが\textbf{割れんばかりに}、大きな声を出している。
        \item 今にも\textbf{泣き出しんばかりに}興奮している。
    \end{itemize}
\end{example}

\separatorline


\subsection{「ばかりだ」表只是}

% 语法结构
\begin{codeblock}
・・・ばかりだ
\end{codeblock}

% 语法解释
表示“光是”,\textbf{“只是”}。通常带有消极评价的语感。

\textbf{「ばかり」前面接动词基本形。}

% 语法示例
\begin{example}
    \begin{itemize}
        \item こんな小さな子供を映画に連れて行っても,退屈する\textbf{ばかりだ}よ。
        \item あの子は,部屋の中で遊ぶ\textbf{ばかり}で,決して外で遊ばない。
    \end{itemize}
\end{example}


\begin{callout}
    第6课曾学过「・・・てばかりいる」,其表示“一直做某事”,有持续做某事的含义。

    \vspace{10pt}
    
    而本课的「・・・ばかりだ」的核心语义在于“只是”,表示限定。
\end{callout}

\begin{callout}
    如果要使用「ている」,可以使用以下两种形式:

    \begin{itemize}
        \item 「・・・てばかりいる」
        \item 「・・・ているばかり」
    \end{itemize}
    
\end{callout}

\separatorline

\subsection{「かのように」表好像・・・似的}

% 语法结构
\begin{codeblock}
・・・かのように、 ~
\end{codeblock}

% 语法解释

表示\textbf{“好像・・・似的”},常与「まるで」搭配使用,整个短句作为副词使用。

\begin{callout}
    本语法可以拆解成「まるで・・・のように」与「か」,其中前者已经学过,表示“好像・・・似的”;后者表示不确定。

    \vspace{10pt}

    \textbf{「か」强化了不确定的语气,使得表达更加委婉,说话人不十分肯定。}
\end{callout}

\textbf{「かのように」前面可以接「动词/一类形容词 普通体」、「形容动词/名词 词干 + である」。}

% 语法示例
\begin{example}
    \begin{itemize}
        \item 彼は、まるで激しい運動をした\textbf{かのように}、息を弾ませていた。
        \item 私が悪い\textbf{かのように}言わないでください。
        \item 彼は病気なのに、会社では健康である\textbf{かのように}振る舞っている。
        \item まるでこの家の主人である\textbf{かのように}、威張っている。
    \end{itemize}
\end{example}

\separatorline

\subsection{「こんな・・・ことはない」表没有比这更・・・的了}

% 语法结构
\begin{codeblock}
こんな・・・ことはない
\end{codeblock}

% 语法解释

表示\textbf{“没有比这更・・・的了”},用于强调某事物的极端性质。

「こと」可以用「人」「話」等词替换,表达出不同的含义。

% 语法示例
\begin{example}
    \begin{itemize}
        \item 母親が死ぬなんて、\textbf{こんな}辛い\textbf{ことはありません}。
        
        (母亲去世了,没什么比这更痛苦的了。)

        \item 友達に裏切られるなんて、\textbf{こんな}不愉快な\textbf{ことはない}。
        \item 昨日約束したことを守らないなんて、\textbf{こんな}ひどい\textbf{話はない}。
        \item \textbf{こんな}立派な\textbf{人はいません}。
        
        (没有比他出色的人了。)
    \end{itemize}
\end{example}

\separatorline

\subsection{「・・・ては~、・・・ては~、」表两组动作重复进行}

% 语法结构
\begin{codeblock}
[・・・て] は [~动词连用形]、 [・・・て] は [~动词连用形]
\end{codeblock}

% 语法解释
表示两组动作重复进行。

\textbf{将「动词て形」+「は」+「动词连用形」的结构重复两次,表示两组动作交替进行,反复发生。}

% 语法示例
\begin{example}
    \begin{itemize}
        \item 休みの日は、\textbf{食べては寝、食べては寝}している。
        
        (休息日里,就是吃了又睡,吃了又睡。)

        \item おもちゃを\textbf{組み立てては壊し、組み立てては壊し}、遊んでいる。
    \end{itemize}
\end{example}

\separatorline

\section{词语与用法}
\subsection{「ようとする」表试图,就要・・・}

% 语法结构
\begin{codeblock}
[・・・よう] とする
\end{codeblock}

% 语法解释

\textbf{「动词意志形+とする」表示“试图・・・”,“就要・・・”,表示某人正要做某事或试图做某事的状态。}

再加上「ている」则表达为了做某个动作而正在进行一定准备。

「ようとしている」后接「と」,表示正处于那种状态时,发生了另一件事。

% 语法示例
\begin{example}
    \begin{itemize}
        \item 出かけ\textbf{ようとしていると}、友達が訪れてきた。 (正要出门时,朋友来了。)

        \item 家へ帰\textbf{ろうとしていると}、課長に呼ばれた。
    \end{itemize}
\end{example}