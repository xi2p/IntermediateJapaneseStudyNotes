\chapter{標準日本語 第19課}

\separatorline

\section{文法}

\subsection{「かねない」表有可能}

% 语法结构
\begin{codeblock}
[・・・动词连用形] かねない
\end{codeblock}

% 语法解释
表示有可能发生\textbf{不好的结果}。“有可能会・・・”,\textbf{“搞不好会・・・”}。

\begin{callout}
    「かねる」表示“不能够・・・”,“无法・・・”,而「かねない」是其否定形,表示“有可能会・・・”。

    \vspace{10pt}

    两者都是较为正式的表达,常用于\textbf{书面语}或正式场合。两者都仅用于描述不好的结果或情况。
\end{callout}

\textbf{「かねない」前面接动词连用形。}

% 语法示例
\begin{example}
    \begin{itemize}
        \item 窓を開けたまま外出したら、泥棒が\textbf{入りかねません}。
        \item 子供をそんなに叱ったら、\textbf{家出しかねません}。
    \end{itemize}
\end{example}

\separatorline
\subsection{「決して・・・ない」表完全否定}

% 语法结构
\begin{codeblock}
決して [・・・ない]
\end{codeblock}

% 语法解释
表示“绝不・・・”,\textbf{“决不・・・”}。

% 语法示例
\begin{example}
    \begin{itemize}
        \item 彼女は、\textbf{決して}コーヒーを\textbf{飲まない}。
        \item 彼は、\textbf{決して}間違ったことを\textbf{言わない}人です。
        
        (他是一个绝不会说错话的人。)
    \end{itemize}
\end{example}

\separatorline

\subsection{「必ずしも・・・ない」表部分否定}

% 语法结构
\begin{codeblock}
必ずしも [・・・ない]
\end{codeblock}

% 语法解释
表示\textbf{部分否定},“不一定・・・”,“未必・・・”。

「とは限らない」也表示“不一定・・・”,“未必・・・”,在一句话中,两者可以一起使用。

% 语法示例
\begin{example}
    \begin{itemize}
        \item \textbf{必ずしも}君の言う通りでは\textbf{ない}。 (不一定完全像你说的那样。)

        \item 彼は今日来る\textbf{とは限らない}。
        \item 彼は\textbf{必ずしも}正しい\textbf{とは限らない}。
    \end{itemize}
\end{example}

\separatorline

\subsection{「ようでは」表条件}

% 语法结构
\begin{codeblock}
・・・ようでは、 ~
\end{codeblock}

% 语法解释
表示条件,\textbf{在该条件下将发生糟糕的事情}。 「ようでは」 = 「よう」 + 「で」 + 「は」。


\textbf{前续规则同「よう」}。

% 语法示例
\begin{example}
    \begin{itemize}
        \item お母さんと一緒でないと困る\textbf{ようでは}、どこへも行けません。
        
        (如果没有妈妈在身边就会感到困扰的话,是哪里也去不了的。)

        \item これぐらいの問題が解けない\textbf{ようでは}、それこそ子供たちに笑われてしまいます。
        \item そんなに嫌いな食べ物が多い\textbf{ようでは}、健康にもよくないですよ。
        \item 京都への旅行が不安な\textbf{ようでは}、留学なんかできませんよ。
    \end{itemize}
\end{example}

\separatorline

\subsection{「にもかかわらず」表逆接}

% 语法结构
\begin{codeblock}
・・・にもかかわらず、 ~
\end{codeblock}

% 语法解释
表示\textbf{逆接},意思类似于\textbf{「のに」},但「にもかかわらず」更正式,用于书面语。

\begin{tabularx}{\linewidth}{|X|X|}
\hline
词性 & 接续规则 \\
\hline
动词 & \textbf{普通体} + にもかかわらず\\
\hline
一类形容词 & \textbf{普通体} + にもかかわらず\\
\hline
形容动词 & \textbf{词干 + である} + にもかかわらず \\
\hline
\multirow{2}{*}{名词} & \textbf{名词} + にもかかわらず \\
\cline{2-2}
& \textbf{名词 + である} + にもかかわらず \\    
\hline

\end{tabularx}

\begin{callout}
    「にもかかわらず」源于「関わる」,其意为“与・・・有关”。
    
    若去掉语法中的「も」则构成「にかかわらず」,表示“不论・・・”,“无论・・・”。
\end{callout}

% 语法示例
\begin{example}
    \begin{itemize}
        \item 雷が鳴っている\textbf{にもかかわらず}、子供たちは外で遊んでいます。
        \item あれほど注意しておいた\textbf{にもかかわらず}、彼は今日も遅刻した。
        \item 私の戒め\textbf{にもかかわらず}、彼はお酒をやめませんでした。
        \item 台風である\textbf{にもかかわらず}、出かけてきました。
    \end{itemize}
\end{example}

\separatorline

\section{词语与用法}
\subsection{「にしても」表让步}

% 语法结构
\begin{codeblock}
・・・にしても、 ~
\end{codeblock}

% 语法解释
表示\textbf{让步},“即使・・・也~”,“・・・做是要做的,但~”。

% 语法示例
\begin{example}
    \begin{itemize}
        \item 先生の家へ行くにしても、お土産も持たないでいくわけには行かないよ。
    \end{itemize}
\end{example}