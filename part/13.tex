\chapter{標準日本語 第13課}

\separatorline

\section{使役被动态活用}

使役被动态是「使役 + 被动」,从名字上就可以看出来,是\textbf{先变成使役态,再变成被动态}。

对于一类动词,还可以直接把使役态的「せる」变成「される」,从而构成使役被动态。\textbf{而且这种形式更常用一些}。

\begin{tabularx}{\linewidth}{|X|X|X|X|X|}
    \hline
    词性 & 活用规律 & 原型示例 & 使役态 & 使役被动态 \\
    \hline
    一类动词 & 未然形 + せられる / \textbf{される} & 書く & 書かせる & 書かせられる / 書かされる \\
    \hline
    二类动词 & 未然形 + させられる & 食べる & 食べさせる & 食べさせられる \\
    \hline
    来る &  来させられる & / & / & / \\
    \hline
    する &  させられる & / & / & / \\
    \hline
\end{tabularx}



\separatorline

\section{文法}

\subsection{「・・・のは、 ・・・からだ」表原因}

% 语法结构
\begin{codeblock}
・・・のは、 ~からだ
\end{codeblock}

% 语法解释
表示\textbf{“・・・的原因是~”}。

\begin{callout}
    这个语法的本质是「のは」+「からだ」,「のは」把前项名词化,作为后项的原因、理由。
\end{callout}

\textbf{「のは」前面用用言连接名词的形式。}

% 语法示例
\begin{example}
    \begin{itemize}
        \item \textbf{遅刻したのは}、列車が遅れたからだ。
        
        (迟到是因为列车晚点了)

        \item 今日、君と\textbf{会えないのは}、山田さんとの約束があるからです。
        \item 私が王さんを\textbf{好きなのは}、彼が親切な人だからです。
    \end{itemize}
\end{example}

\separatorline
\subsection{「てばかりはいられない」表不能只做某事}

% 语法结构
\begin{codeblock}
[・・・て] ばかりはいられない
\end{codeblock}

% 语法解释
表示不能全面地做某事,\textbf{不能只做某事而不做其他事情}。

% 语法示例
\begin{example}
    \begin{itemize}
        \item いつまでも\textbf{悲しんでばかりはいられない}。
        
        (一直悲伤着也不行)

        \item 父が入院したので、今までのように\textbf{旅行してばかりはいられない}。
    \end{itemize}
\end{example}

\textbf{「てはいられない」本身表示“不能这样做,必须做别的事”。}加上副词「ばかり」后才表示“不能只做某事”。

\begin{example}
    \begin{itemize}
        \item もうすぐ試験なので、\textbf{遊んではいられない}。
        \item のんびりお茶を\textbf{飲んではいられない}。
    \end{itemize}
\end{example}

\separatorline

\subsection{「にちがいない」表肯定}

% 语法结构
\begin{codeblock}
・・・にちがいない
\end{codeblock}

% 语法解释
表示\textbf{说话人判断}可能性很大,相当于“一定,肯定”。

\begin{callout}
    「にちがいない」的本质是\textbf{「に」+「違い」+「ない」}。
    
    \vspace{10pt}

    其中「に」提示对象,「違いない」表示“没有错误”,用来强调说话人的判断是正确的。
\end{callout}

「に」前面接续规则如下:

\begin{tabularx}{\linewidth}{|X|X|}
    \hline
    词性 & 接续规则 \\
    \hline
    动词 & 普通体 + にちがいない \\
    \hline
    一类形容词 & 普通体 + にちがいない \\
    \hline
    形容动词 & \textbf{词干 + (だった / ではない)} + にちがいない\\
    \hline
    名词 & \textbf{名词 + (だ / ではない)} + にちがいない \\
    \hline
\end{tabularx}

% 语法示例
\begin{example}
    \begin{itemize}
        \item あなたが誘えば、田中さんも\textbf{行くに違いない}。
        
        (你要是邀请的话,田中一定会去的。)

        \item 鈴木さんの娘だから、背が\textbf{高いに違いない}。
        \item 手紙をくれたのは、\textbf{彼に違いない}。
        \item お正月はみんな故郷に帰るので、東京は\textbf{静かに違いない}。
    \end{itemize}
\end{example}

\separatorline

\subsection{「とともに」表同时}

% 语法结构
\begin{codeblock}
・・・とともに、 ~
\end{codeblock}

% 语法解释
表示\textbf{“・・・的同时,~”}。

\textbf{「と」前面用动词的基本形}。

% 语法示例
\begin{example}
    \begin{itemize}
        \item 家を\textbf{買うとともに}、自動車も買った。
        
        (买了房子,同时也买了汽车。)

        \item 彼は小説を\textbf{書くとともに}、絵もかいている。
    \end{itemize}
\end{example}

\separatorline

\subsection{使役被动态}

% 语法解释
可以用使役被动态表达\textbf{“被强制进行某项动作”},多带有“硬让做不愿意做的事情”的意思。

% 语法示例
\begin{example}
    \begin{itemize}
        \item 私は嫌いな野菜を\textbf{食べさせられました}。
        
        (我被强迫吃了讨厌的蔬菜。)

        \item トイレの掃除を\textbf{させられました}。
        \item 母の代わりに買い物に\textbf{行かされた}。
    \end{itemize}
\end{example}

「考えさせられる」不带有强制的意思,而表示自发的,受到触发而情不自禁地想某事。

\begin{example}
    \begin{itemize}
        \item 彼の話を聞いて、いろいろ\textbf{考えさせられました}。
        
        (听了他的话,不禁想了很多。)
    \end{itemize}
\end{example}

\separatorline

\section{词语与用法}
\subsection{「だって」表举例、也}

% 语法结构
\begin{codeblock}
・・・だって~
\end{codeblock}

% 语法解释
表示\textbf{在同类事物中举例},类似于汉语的“也”,类似于「も」。但「だって」只用于口语。

% 语法示例
\begin{example}
    \begin{itemize}
        \item おじさん\textbf{だって}、おばさん\textbf{だって}行くんだから、私も行きたいわ。
        \item カルチャーセンターには、語学の講座\textbf{だって}あるよ。
        
        (文化中心也有语言课程哦。)

    \end{itemize}
\end{example}