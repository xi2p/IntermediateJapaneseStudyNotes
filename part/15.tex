\chapter{標準日本語 第15課}

\separatorline

\section{文法}

\subsection{「のも無理はない」表理所当然,不得已}

% 语法结构
\begin{codeblock}
・・・のも無理はない

・・・のも当然だ
\end{codeblock}

% 语法解释
是说话人的评价,表示某事\textbf{理所当然},或者是\textbf{不得已}的。

\textbf{「の」的作用是将前面的句子名词化},其前面接续规则由此语法决定。

% 语法示例
\begin{example}
    \begin{itemize}
        \item 2時間も遅刻したのだから、みんなが怒る\textbf{のも無理はない}。
        
        (因为迟到了2个小时,大家生气也是\textbf{理所当然}的。)

        \item こんなに似ているのだから、間違える\textbf{のも無理はない}。
        \item もう大人なのだから、自分のやったことに責任を持つ\textbf{のも当然だ}。
        \item 技術は発達するので、ここが便利\textbf{なのも無理はない}。
    \end{itemize}
\end{example}

\separatorline
\subsection{「にかけて」表关于}

% 语法结构
\begin{codeblock}
・・・にかけて(は)、 ~
\end{codeblock}

% 语法解释
表示\textbf{“关于・・・”“在・・・这一点上”}。

\textbf{「にかけて」的前面接名词}。

% 语法示例
\begin{example}
    \begin{itemize}
        \item 飛行機\textbf{にかけては}、彼がクラスで一番よく知っている。
        \item 走ること\textbf{にかけては}、彼女は誰にも負けない。
    \end{itemize}
\end{example}

\separatorline

\subsection{「わけである / わけではない」表逻辑推导}

% 语法结构
\begin{codeblock}
・・・わけである

・・・わけではない
\end{codeblock}

% 语法解释

\begin{callout}
「わけ」本身表示“理由”。在日语发展过程中,“わけ”逐渐语法化为一个形式名词,\textbf{用于表达逻辑推理、解释说明的语气}。

\vspace{10pt}

其语义核心始终围绕“理据性”,即从事理、逻辑上进行说明或推断。
\end{callout}

「わけである」「わけだ」「わけです」用于句末,表示从事物发展趋势及情理上看,这种情况是\textbf{理所当然,合乎逻辑}。

% 语法示例
\begin{example}
    \begin{itemize}
        \item こんなに熱があるのだから、気分が悪い\textbf{わけです}。
        
        (发这么高的烧,感觉不舒服是理所当然的。)

        \item 彼は家を建てました。つまりやっと長い間の希望がかなった\textbf{わけです}。
        
        (他盖了房子。也就是说,长期以来的愿望终于实现了。)

        \item 中国に5年いたから、中国語が上手\textbf{なわけだ}。
        \item 先日掃除をしたから、きれいだった\textbf{わけです}。
    \end{itemize}
\end{example}

「わけではない」表示\textbf{“并非・・・”“不一定・・・”},并非理所当然。

\begin{example}
    \begin{itemize}
        \item 足をけがしていますが、歩けない\textbf{わけではありません}。
        
        (虽然脚受伤了,但并非不能走路。)

        \item 彼は笑っているからと言って、許してくれた\textbf{わけではない}。
    \end{itemize}
\end{example}

\separatorline

\section{词语与用法}
\subsection{「において」表在・・・上}

% 语法结构
\begin{codeblock}
・・・において~
\end{codeblock}

% 语法解释
表示\textbf{“在・・・上”“在・・・方面”}。

% 语法示例
\begin{example}
    \begin{itemize}
        \item この決定は、会議\textbf{において}発表された。
        
        (这个决定\textbf{在}会议\textbf{上}宣布了。)

        \item この製品は、価格\textbf{において}も品質\textbf{において}も優れている。
        
        (这个产品\textbf{在}价格和质量\textbf{上}都很优秀。)

    \end{itemize}
\end{example}

\separatorline

\subsection{「~や~といった」表例示}

% 语法结构
\begin{codeblock}
~や~といった
\end{codeblock}

% 语法解释
\textbf{类似于「~や~のような」},表示例示,“像・・・这样的・・・”。

% 语法示例
\begin{example}
    \begin{itemize}
        \item デパート\textbf{や}スーパー\textbf{といった}大きな店は、まだこの街にありません。
    \end{itemize}
\end{example}

\separatorline

\subsection{「~から~にいたるまで」表范围}

% 语法解释
类似于「~から~まで」,表示范围。\textbf{使用「いたる」来强调范围内的一切},着重表示范围之广,程度之深。

% 语法示例
\begin{example}
    \begin{itemize}
        \item 掃除\textbf{から}子供の世話\textbf{にいたるまで}、彼女は一人で全部やっている。
        
        (从打扫卫生到照顾孩子,\textbf{全都}是她一个人做的。)

        \item 北海道\textbf{から}沖縄\textbf{にいたるまで}、日本中雨が降っているそうだ。
    \end{itemize}
\end{example}

\separatorline

\subsection{「せい」表消极的原因}

% 语法解释
\textbf{表示事情发生的原因}。这种事情通常是消极的、不好的。包含了不满、责备等情绪。

常用表达形式有「~のせいだ」「~のせいで」「~せいか」等。

「せい」是名词性质,前面接续规则与名词相同。

% 语法示例
\begin{example}
    \begin{itemize}
        \item ハイキングが延期になったのは\textbf{雨のせいだ}。
        \item \textbf{雨のせいで}、ハイキングが延期になった。
        \item 電車が遅れた\textbf{せいで}、遅刻してしまった。
        \item 操作は複雑\textbf{なせいで}、誰もこの機械を使わない。
    \end{itemize}
\end{example}

\separatorline