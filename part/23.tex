\chapter{標準日本語 第23課}

\separatorline

\section{文法}

\subsection{「をはじめとして」表以・・・为首}

% 语法结构
\begin{codeblock}
・・・をはじめとして、 ~
\end{codeblock}

% 语法解释

表示\textbf{“以・・・为首”}。可以体现出主次关系,也可以表达出时间关系。

「をはじめ」是其口语形式。

% 语法示例
\begin{example}
    \begin{itemize}
        \item 社長\textbf{をはじめとして}、社員全員が式に出席します。
        
        (以社长为首,公司全员都在仪式上出席了)

        \item 東京\textbf{をはじめとして}、名古屋や大阪などの大都市では交通事故が多い。
        \item 駅前に高層ビルが建ったの\textbf{をはじめとして}、次々に高いビルが建った。
        
        (以车站前建起了高层大楼为首,接连不断地建起了高楼大厦)
    \end{itemize}
\end{example}

\separatorline
\subsection{「までもない」表用不着・・・}

% 语法结构
\begin{codeblock}
・・・までもない
\end{codeblock}

% 语法解释
「までもない」与「ことはない」一样,表示“不必・・・”,\textbf{“用不着・・・”}。相当于「・・・する必要はない」。

\textbf{「まで」前面接动词基本形。}

% 语法示例
\begin{example}
    \begin{itemize}
        \item もうみんなが知っていることだから、わざわざ説明する\textbf{までもない}。
        \item 誰が嘘をついているかは明らかで、考える\textbf{までもない}。
    \end{itemize}
\end{example}

\separatorline

\subsection{「からには」表既然}

% 语法结构
\begin{codeblock}
・・・からには、 ~
\end{codeblock}

% 语法解释
表示\textbf{“既然・・・就・・・”}。用于表示说话人基于某种前提,做出自认为是理所当然的判断或决定。\textbf{和之前学过的「以上」意思相同。}

\textbf{「から」前面接「动词/一类形容词 普通体」,或接「名词/形容动词 + である」。}

% 语法示例
\begin{example}
    \begin{itemize}
        \item 約束をした\textbf{からには}、最後まで責任を持ってやってほしい。
        \item 教師である\textbf{からには}、このぐらいのことは知っておくべきだ。
    \end{itemize}
\end{example}

\separatorline

\subsection{「ないでもない」双重否定表肯定}

% 语法结构
\begin{codeblock}
[・・・ない型] でもない
\end{codeblock}

% 语法解释
\textbf{双重否定表肯定},表示“也不是不・・・”,“也可以・・・”。

此语法\textbf{不用于描述积极肯定的情况},表示「少し・・・する」「・・・する可能性がある」的意思。

% 语法示例
\begin{example}
    \begin{itemize}
        \item 洋子さんが行くなら、私も行か\textbf{ないでもないんです}が。
        
        (如果洋子去的话,我也不是不能去。)(存在可能性)

        \item こうなることは予期してい\textbf{ないでもなかった}。
        
        (这种事也不是没有预料到。)(有一点预料到)
    \end{itemize}
\end{example}

\separatorline


\section{词语与用法}
\subsection{「だらけ」表表面全是・・・}

% 语法结构
\begin{codeblock}
・・・だらけ
\end{codeblock}

% 语法解释
表示\textbf{“某物的表面净是・・・”},多用于表示不好的东西。

% 语法示例
\begin{example}
    \begin{itemize}
        \item パンがカビ\textbf{だらけ}になった。
        
        (面包上全是霉。)

        \item 転んで、服が泥\textbf{だらけ}になってしまった。
    \end{itemize}
\end{example}