\chapter{標準日本語 第12課}

课文主题:\textbf{信件}

\separatorline

\section{文法}

\subsection{「かと思うと」表两件事连续发生}

% 语法结构
\begin{codeblock}
・・・かと思うと、 (たちまち)~

・・・かと思ったら、 ~
\end{codeblock}

% 语法解释
描写\textbf{两件事连续发生},可以使用副词「たちまち」「すぐ」「もう」等强调两件事之间时间很短。

\textbf{「か」前面用动词的「た」形。}

\begin{callout}
    这个语法实际上是「・・・か」+「と思う」+「と / たら」,表示“刚刚觉得・・・就~”。「と / たら」表示“一・・・就~”。

    \vspace{10pt}

    由于使用了「・・・か」,所以此语法暗含一种说话人不确定前项有没有发生的语气,事发突然,有恍惚感。
\end{callout}

% 语法示例
\begin{example}
    \begin{itemize}
        \item 雷が\textbf{鳴ったかと思うと}、たちまち雨が降り出した。
        
        (刚打雷,雨就下起来了。)

        \item 信号が青に\textbf{なったかと思うと}、すぐに赤に変わった。
        \item さっき\textbf{出かけたかと思ったら}、すぐ帰って来た。
    \end{itemize}
\end{example}

\begin{callout}
    据笔者所在的一个群组内的在日工作多年的群友分享,这个语法并不常用,甚至说是“没见到过有人用”。本语法偏向书面、文学语言。

    \vspace{10pt}

    考虑到本教材是上世纪80年代编写的,语法内容有些过时也是情有可原的。

\end{callout}

\separatorline

\subsection{「おかげで」表多亏了}

% 语法结构
\begin{codeblock}
・・・おかげで、 ~
\end{codeblock}

% 语法解释
表示因果原因,前面是原因,后面是结果。此语法中,结果多为积极的内容,相当于\textbf{“多亏了・・・”}。本语法常用于表达感谢之情。

\textbf{「おかげ」可以视为名词,前接续规则同名词。}

前接动词时,常常会使用「てくれる」「ていただく」的形式。但也可以用纯普通体。


% 语法示例
\begin{example}
    \begin{itemize}
        \item 彼は\textbf{手伝ってくれたおかげで}、仕事が早く済んだ。
        
        (多亏了他的帮忙,工作很快就完成了。)

        \item 彼は食材を買っておいた\textbf{おかげで}、料理がすぐできた。
        \item この地方は、夏が\textbf{涼しいおかげで}、観光客がたくさん来ます。
        \item 看護婦さんが\textbf{親切なおかげで}、入院生活は楽しいです。
        \item \textbf{彼のおかげで}、列車の切符が買えました。
    \end{itemize}
\end{example}

\separatorline

\subsection{「ことと存じます」表推测、判断}

% 语法结构
\begin{codeblock}
・・・ことと存じます
\end{codeblock}

% 语法解释
「存じます」是「思います」的自谦语。

\textbf{「ことと存じます」、「ことと思います」表示推测、判断},作为\textbf{书信}习惯用语常常使用。

「こと」前接续规则如下:

\begin{tabularx}{\linewidth}{|X|X|}
    \hline
    词性 & 接续规则 \\
    \hline
    动词 & 普通体 + こと \\
    \hline
    一类形容词 & 普通体 + こと \\
    \hline
    形容动词 & \textbf{词干 + な / だった} + こと\\
    \hline
\end{tabularx}

% 语法示例
\begin{example}
    \begin{itemize}
        \item 新しい仕事を始められて、まいにち\textbf{おいそがしいことと存じます}。

        \item 東京の残暑は\textbf{大変なことと存じます}。
        
        (想来东京秋老虎的酷热一定很难熬吧。)

        \item 誰もいなくて、\textbf{不安だったことと存じます}。
    \end{itemize}
\end{example}

\separatorline

\subsection{「せっかく・・・のだから」表(好不容易)原因}

% 语法结构
\begin{codeblock}
せっかく・・・のだから、 ~
\end{codeblock}

% 语法解释
「せっかく」表示“好不容易”、「・・・のだから」表示原因,后句常叙述对听话人的劝诱。

\begin{callout}
    此语法可以拆分成「せっかく」+「・・・のだ」+「から」。

    \vspace{10pt}

    其中,「・・・のだ」表示解释说明,「から」表示原因。

    \vspace{10pt}

    因此\textbf{「・・・」处的接续规则应与「・・・のだ」的接续规则相同。}
    
    且可以发声音变,有「せっかく・・・んだから」。
\end{callout}

% 语法示例
\begin{example}
    \begin{itemize}
        \item \textbf{せっかく}歌舞伎の切符を\textbf{もらったのだから}、今度の日曜日に見に行こう。
        
        (好不容易得到了歌舞伎的票,这个星期天去看吧。)

        \item \textbf{せっかく}留学\textbf{したんだから}、しっかり勉強したいと思います。
    \end{itemize}
\end{example}

使用「せっかく・・・のに」可以表示逆接。「・・・」处的接续规则符合「・・・のに」的接续规则。

\begin{example}
    \begin{itemize}
        \item \textbf{せっかく}料理を作って\textbf{待っていたのに}、息子は帰ってこなかった。
        
        (好不容易做了饭等着,儿子却没回来。)

        \item \textbf{せっかく}教えて\textbf{もらったのに}、忘れてしまいました。
    \end{itemize}
\end{example}
