\chapter{標準日本語 第16課}

\separatorline

\section{文法}

\subsection{「となると」表条件假设}

% 语法结构
\begin{codeblock}
・・・となると、 ~
\end{codeblock}

% 语法解释
假设・・・的动作或状态,引出后面情况。类似于「とすると」。

「となると」前接续规则如下:

\begin{tabularx}{\linewidth}{|X|X|}
    \hline
    词类 & 接续 \\
    \hline
    动词 & 普通体 + となると \\
    \hline
    一类形容词 & 普通体 + となると \\
    \hline
    \multirow{2}{*}{形容动词} & 词干 + となると \\
    \cline{2-2}
     & 词干 + だ / である / でない + となると \\
    \hline
    \multirow{2}{*}{名词} & 词干 + となると \\
    \cline{2-2}
     & 词干 + だ / である / でない + となると \\
    \hline
\end{tabularx}

% 语法示例
\begin{example}
    \begin{itemize}
        \item 新しい車を買うとなると、200万円くらいは必要だ。
        
        (如果要买新车的话,大概需要200万日元。)

        \item 昨日貸した本が難しいとなると、今度の本は読めないよ。
        \item 山田くんが病気となると、他に手伝ってくれる人を探さなくてはならなくなる。
        \item この店の料理も好きでないとなると、もう案内できる店はありません。
    \end{itemize}
\end{example}

\newpage

\separatorline
\subsection{「限り」表时间范围}

% 语法结构
\begin{codeblock}
・・・限り、 ~
\end{codeblock}

% 语法解释
表示在「限り」前面所述・・・事情的持续期间内~。

「限り」可以用「間」替换,但「限り」所表现说话人的心情更强烈一些。

「限り」前面的接续规则如下:

\begin{tabularx}{\linewidth}{|X|X|}
    \hline
    词类 & 接续 \\
    \hline
    动词 & 普通体 + 限り \\
    \hline
    一类形容词 & 普通体 + 限り \\
    \hline
    形容动词 & 词干 + である / でない + 限り \\
    \hline
    名词 & 词干 + である / でない + 限り \\
    \hline
\end{tabularx}

% 语法示例
\begin{example}
    \begin{itemize}
        \item 君が一緒にいる限り、安心していられるよ。
        
        (只要你在我身边,我就能安心。)

        \item 両親と一緒に暮らしている限り、自由になれない。
        \item あなたが私の友達である限り、何でもしてあげよう。
        \item この機械が巨大である限り、実用化は不可能だ。
    \end{itemize}
\end{example}

\separatorline

\subsection{「とされている」表公认事实}

% 语法结构
\begin{codeblock}
・・・とされている
\end{codeblock}

% 语法解释
表示一般所公认的事实。「と」的前面接普通体。

% 语法示例
\begin{example}
    \begin{itemize}
        \item 今度の選挙は、田中氏が当選するとされている。
        
        (这次选举,大家都认为田中先生会当选。)

        \item ここは地震が多いので、高い建物は建てられないとされている。
        \item 冬には、あの山に登るのは不可能だとされている。
    \end{itemize}
\end{example}

\separatorline

\newpage

\subsection{「やいなや」表事情连续发生}

% 语法结构
\begin{codeblock}
・・・やいなや、 (たちまち)~
\end{codeblock}

% 语法解释
与「かと思うと」意思几乎相同,表示前后两件事情紧接着发生。

\begin{callout}
    「や」是古典日语中表示疑问的助词,相当于「か」。「いな」是古典日语中表示否定的助词,相当于「ない」。

    \vspace{10pt}

    因此,「やいなや」相当于现代日语中的「かないか」,和「かと思うと」的语义内核是一样的。
\end{callout}

% 语法示例
\begin{example}
    \begin{itemize}
        \item 家に帰ってくるやいなや、大きな声で泣き出した。
        
        (刚一回到家,就大声哭了起来。)
    \end{itemize}
\end{example}

\separatorline

\section{词语与用法}

\subsection{「と言うのは」作发语词,解释根源}

% 语法结构
\begin{codeblock}
と言うのは、 ~
\end{codeblock}

% 语法解释
把「と言うのは」放在句首,表示解释前面所说的话的根源、理由等。句末要接「から」「のだ」等表示解释、理由的表达。

% 语法示例
\begin{example}
    \begin{itemize}
        \item この地方にはせの高い木が少ない。と言うのは、風が強いからだ。
        
        (这个地方高大的树木很少。那是因为风很大。)
    \end{itemize}
\end{example}