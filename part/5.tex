\chapter{標準日本語 第5課}

摘要:\textbf{無し}

\separatorline

\section{文法}

\subsection{「てたまらない」表无法忍受的地步}

% 语法结构
\begin{codeblock}
[・・・一类形て] たまらない

[・・・二类形で] たまらない
\end{codeblock}

% 语法解释
表示某种感觉或状态\textbf{强烈到无法忍受}的程度,相当于\textbf{“・・・得受不了” “・・・得不得了”}。

\begin{callout}
    「てたまらない」的本质是「て」+「たまらない」。而「たまらない」是「たまる」的否定形式,表示“无法忍受”。 
    因此,「てたまらない」\textbf{前接一类形容词/二类形容词「て」形}。
\end{callout}

% 语法示例
\begin{example}
    \begin{itemize}
        \item 自分の子供が\textbf{可愛くてたまらない}。
        
        (自己的孩子可爱得不得了。)

        \item \textbf{寒くてたまらない}ので、家の中にいました。
        
        \item 私は\textbf{不安でたまらない}。
        
    \end{itemize}
\end{example}

\separatorline

\subsection{一类形容词的连用中止}

% 语法结构
\begin{codeblock}
一类形容词\textbf{词干} + く
\end{codeblock}

% 语法解释
和此前学的动词一样,被“连用中止”的句子间可以有多种关系,有: \textbf{并列 ・ 对比 ・ 原因}

% 语法示例

\begin{example}
    \begin{itemize}
        \item 田中君は頭が\textbf{良く}、運動もできる。 (并列)
        \item 妹は目が\textbf{大きく}、姉は鼻が高い。  (对比)
        \item 外が\textbf{うるさく}、眠れない。     (原因)
    \end{itemize}
\end{example}

「て」形作中止作用,也能表达上述关系,但\textbf{连用中止更正式,更书面化},多用于书面语。

\separatorline

\subsection{「にしたがって」表随着}

% 语法结构
\begin{codeblock}
・・・にしたがって、 ~
\end{codeblock}

% 语法解释
相当于\textbf{“随着・・・,~”},表示随着某种变化,另一种变化也随之发生。

\textbf{「にしたがって」前面接动词基本形。}

\begin{callout}
    初级阶段还学过「にしたがって」的另一种用法,表示“按照・・・”。
\end{callout}

% 语法示例
\begin{example}
    \begin{itemize}
        \item 夏休みが近づく\textbf{にしたがって}、嬉しさが増してきた。
        
        (\textbf{随着}暑假的临近,快乐感增加了。)

        \item 有名になる\textbf{にしたがって}、仕事が増えた。
        
        \item 日本の習慣\textbf{にしたがって}、「いただきます」と言います。  (表按照)
    \end{itemize}
\end{example}

\separatorline


\section{词语与用法}
\subsection{日常会话中的一些说法}

\subsubsection{じゃ}

\textbf{是「では」的口语形式},常用于日常会话中。除了引出话题,「では」还可以表示“在・・・情况下”的意思。(「で」+「は」)

\begin{example}
    \begin{itemize}
        \item 歩いてきたん\textbf{じゃ}、疲れただろう。 \textbf{(这里「ん」是「の」的口语形式)}
        
        (走着来的话,肯定累了吧。)

        \item 私は学生\textbf{じゃ}ありません。

        \item \textbf{じゃ}、始めましょうか。
    \end{itemize}
\end{example}

\subsubsection{・・・かい}

和「か」作用相同,表示疑问,但更口语化,男性用语。

\begin{example}
    \begin{itemize}
        \item 何をしている\textbf{かい}。

        \item 東京より京都のほうが寒い\textbf{かい}。
        
    \end{itemize}
\end{example}

\subsubsection{・・・って}

\textbf{是「と」的口语形式},引出言语或想法的内容。

\begin{example}
    \begin{itemize}
        \item 彼は来ない\textbf{って}言いました。
    \end{itemize}
\end{example}

