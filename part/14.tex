\chapter{標準日本語 第14課}

摘要:\textbf{無し}

\separatorline

\section{文法}

\subsection{「ことから」表线索}

% 语法结构
\begin{codeblock}
・・・ことから、 ~
\end{codeblock}

% 语法解释
以前面的事情为线索,引出后续的结论、成果。

% 语法示例
\begin{example}
    \begin{itemize}
        \item 二人の顔が似ている\textbf{ことから}、親子ではないかと思った。
        \item たくさんの人が集まっている\textbf{ことから}、何か事件が起こったと感じた。
    \end{itemize}
\end{example}

\separatorline
\subsection{「ばかりでなく」表不仅・・・而且・・・}

% 语法结构
\begin{codeblock}
・・・ばかりでなく、 ~
\end{codeblock}

% 语法解释
表示\textbf{“不仅・・・而且~”}。

「ばかりでなく」前面可以接\textbf{动词普通体,名词}。

在后句中,可以使用助词\textbf{も}、\textbf{まで}等。其中「まで」表示程度之甚。

% 语法示例
\begin{example}
    \begin{itemize}
        \item あの歌手は若者に人気が\textbf{あるばかりでなく}、子供たちに\textbf{も}人気がある。
        \item 最近は、\textbf{女性ばかりでなく}、男性\textbf{まで}赤い服を着ることがあります。
    \end{itemize}
\end{example}

「だけでなく」也可以表达同样的意思。

\begin{callout}
    前几节课学过的「・・・ばかりか、・・・も~」也可以表达类似的意思。
\end{callout}

\separatorline

\subsection{「にしては」表与预想矛盾}

% 语法结构
\begin{codeblock}
・・・にしては、 ~
\end{codeblock}

% 语法解释
类似于「わりには」「のに」,表示\textbf{“与预想相反”},逆接。

% 语法示例
\begin{example}
    \begin{itemize}
        \item 彼は旅行会社に勤めて\textbf{いるにしては}、あまり英語が得意でない。
        \item 彼は\textbf{高校生にしては}、背が低い。
    \end{itemize}
\end{example}
