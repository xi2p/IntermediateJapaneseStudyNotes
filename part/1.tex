\chapter{標準日本語 第1課}

摘要:\textbf{無し}

\separatorline

\section{文法}

\subsection{「として」表作为}

\begin{codeblock}
[~名词]として ・・・
\end{codeblock}

表示对句中的主题/人/事物的某一方面着重叙述时使用。相当于中文的\textbf{“作为・・・”}。

\begin{example}
    \begin{itemize}
        \item 私は学生\textbf{として}日本へ来ました。
        \item 彼は医者\textbf{として}働いています。
        \item あの人は新聞記者\textbf{として}有名です。
    \end{itemize}
\end{example}

可以使用「名词+として\textbf{の}」这个整体\textbf{修饰名词}。

\begin{example}
    \begin{itemize}
        \item 留学生\textbf{としての}王さんは優秀です。
        
        \textbf{作为留学生的}王同学很优秀。
    \end{itemize}
\end{example}

可以使用「名词+として\textbf{は}」表示\textbf{对比}。

\begin{example}
    \begin{itemize}
        \item 王さんは留学生\textbf{としては}優秀です。
        
        王同学作为留学生来说很优秀。(但在其他方面可能不怎么样)
    \end{itemize}
\end{example}

此外,「として」还可以表示\textbf{强调}。

\begin{example}
    \begin{itemize}
        \item 私\textbf{として}は田中さんの意見に賛成です。(\textbf{我个人}赞成田中先生的意见。)
        \item 田中さん\textbf{としての}意見を聞かせてください。
    \end{itemize}
\end{example}

\separatorline

\newpage

\subsection{「うちに / うちは」表动作发生的时间点(段)}

\begin{codeblock}
・・・うちに/ うちは、 ~
\end{codeblock}

\textbf{表示在・・・的期间内,~。}

「うち」可以视为名词,其前接续方式与名词一致,常用:

\begin{itemize}
    \item 动词「ている」形
    \item 一类形容词
    \item 形容动词 + な
    \item 名词 + の (使用表示期间或身份的名词,如「夏休み」、「子供」等)
\end{itemize}

「うちに」表示在期间内\textbf{发生的事情}。

\begin{example}
    \begin{itemize}
        \item 勉強している\textbf{うちに}、 友達がたくさんできました。
        
        在学习的期间内,交了很多朋友。

        \item 暗い\textbf{うちに}、 出発しましょう。
        
        天还没亮的时候,出发吧。

        \item 子供の\textbf{うちに}、 外国語を 習ったほうが いいです。
        
        趁着还是小孩子的时候,学外语比较好。
    \end{itemize}
\end{example}

「うちは」表示在期间内\textbf{持续的状态}。

\begin{example}
    \begin{itemize}
        \item 風が吹いている\textbf{うちは}、外に出ることができません。
        
        风还在吹的时候,不能出去。

        \item 朝の\textbf{うちは}まだ涼しいですが、 11時を過ぎると暑くなります。
        
        早晨的时候还凉快,过了11点就热起来了。
    \end{itemize}
\end{example}

\separatorline

\subsection{「まで(に)」表动作发生的期限}

\begin{codeblock}
・・・まで(に)、 ~
\end{codeblock}

「まで」前面接表示时间的词语,表示期限。后面是在这一期限以前发生的事情。

「まで」后面是在期限以前持续的状态或动作\textbf{(在全区间内进行)}。「までに」后面是在期限以前发生的动作(\textbf{在某一时间点上进行)}。

\begin{example}
    \begin{itemize}
        \item 3時\textbf{まで}手紙を書きました。
        
        写信\textbf{一直写到}三点钟。

        \item 3時\textbf{までに}手紙を書きました。
        
        \textbf{三点钟以前}写完了信。
    \end{itemize}
\end{example}

\separatorline

\subsection{「(さ)せてもらう/いただく」表请求许可}

\begin{codeblock}
・・・(さ)せてもらう

・・・(さ)せていただく
\end{codeblock}

表示说话人希望得到听话人认可。

\begin{callout}
    这个语法的结构是:动词的使役形 + もらう/いただく。

    \vspace{10pt}

    可以拆开来看这两部分。首先是使役态表示“让・・・做・・・”,然后再用「てもらう」表示把这个动作给我。两者结合就是\textbf{“让我做・・・”}的意思。

    \vspace{10pt}

    「いただく」是「もらう」的敬语。因为使用了「もらう」,所以仍要保持“内は 外に”。(可以省略不说,但要知道)

    \vspace{10pt}

    使用此语法时也要注意时态,若是用过去时,则表示\textbf{“得以・・・了”}(请求成功,动作已发生);若是用现在时,则表示“请求・・・”。
\end{callout}

\begin{example}
    \begin{itemize}
        \item 仕事を\textbf{休ませてもらいました}。
        
        我请假休息了。

        \item 私は本を\textbf{読ませていただきました}。
        
        我\textbf{得以}读了这本书。
    \end{itemize}
\end{example}


\subsection{「(さ)せてください」表请求允许}

\begin{codeblock}
・・・(さ)せてください
\end{codeblock}

表示\textbf{请求}。说话人想要办某事而请求听话人允许。

\begin{example}
    \begin{itemize}
        \item ぜひ読んで、感想を\textbf{聞かせてください}。
        
        请一定读一读,然后\textbf{让我听听}你的感想。(允许我做“听感想”这件事)

        \item 疲れたから少し\textbf{休ませてください}。
        
        \item 約束があるので、今日はもう\textbf{帰らせてください}。

    \end{itemize}
\end{example}

\separatorline

\newpage

\section{词语与用法}

\subsection{「やってくる」}

表示来,和「来る」意思相同,但是还含有涉及这一动作的\textbf{过程}的意思。

\begin{example}
    \begin{itemize}
        \item スミスさんが手を振りながら、\textbf{やって来ました}。
        
        史密斯先生一边挥手一边走过来了。(强调了“走过来”的过程)
    \end{itemize}
\end{example}

\separatorline

\subsection{「ばかり」表只、仅仅}

表示限定,类似于「だけ」,相当于中文的\textbf{“只、仅仅・・・”}。

但两者有不同:「だけ」表示严格排除其他事物,而「ばかり」带有\textbf{被限定事物反复出现}的意思。

\begin{example}
    \begin{itemize}
        \item あの人は、いつも怒って\textbf{ばかりいる}。
        
        那个人\textbf{总是}生气个\textbf{不停}。(表示反复出现的状态)

        \item 彼は遊ぶ\textbf{だけだ}。
        
        他只是玩而已。(排除其他可能性)
    \end{itemize}
\end{example}

\separatorline

\subsection{「のような」表以・・・为例}

\begin{codeblock}
・・・のような ~
\end{codeblock}

表示\textbf{举例},即从众多事物中举出一个或多个例子。相当于中文的\textbf{“像・・・这样的”}。

\begin{example}
    \begin{itemize}
        \item 犬や猫\textbf{のような}動物は嫌いです。

        (我讨厌\textbf{像狗、猫这样的}动物。)
    \end{itemize}
\end{example}

\separatorline

\subsection{「てもよろしいでしょうか」表请求}

\begin{codeblock}
・・・て(も)よろしいでしょうか
\end{codeblock}

类似于「てもいいですか」,\textbf{表示“可以・・・吗?”,即请求。}

使用「よろしい」比「いい」更郑重,用「でしょうか」比「ですか」更委婉。

\textbf{助词「も」可以省略不说。}

\begin{example}
    \begin{itemize}
        \item 帰って\textbf{よろしいでしょうか}。
    \end{itemize}
\end{example}

