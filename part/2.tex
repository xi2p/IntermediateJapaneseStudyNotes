\chapter{標準日本語 第2課}

摘要:\textbf{無し}

\separatorline

\section{文法}

\subsection{「たらどうですか」表建议、劝诱}

\begin{codeblock}
[・・・动词た] らどうですか

[・・・动词た] らいかがですか
\end{codeblock}

表示\textbf{建议}对方进行某种行为。是「・・・たほうがいいです」的更加婉转的表达方式。

「いかが」是「どう」的敬语形式,向长辈或上司提出建议时使用。

\begin{example}
    \begin{itemize}
        \item ここへ遊びに\textbf{来たらどうですか}。
        
        要不要来这里玩玩?

        \item 音楽を聞きながら、お待ちに\textbf{なったらいかがですか}。
        
        要不要一边听音乐一边等呢?
    \end{itemize}
\end{example}

「たらどうですか」有时含有\textbf{责备}的意思,责怪对方没做某事。

\begin{example}
    \begin{itemize}
        \item もう少し\textbf{勉強したらどうですか}。
        
        \sout{阿嫲:阿伟你又在玩电动哦,去看个书好不好。}
    \end{itemize}
\end{example}

\separatorline

\newpage

\subsection{「ような気がする」表推测}

\begin{codeblock}
・・・ような気がする
\end{codeblock}

表示\textbf{推测},说话者没有很大把握,\textbf{不确定度比「よう」更高}。

\begin{callout}
此语法是「よう」+「気がする」的结合体,「よう」变连体形「ような」,修饰「気」。

\vspace{10pt}

「気がする」单独使用时,表示“觉得,好像・・・”。而「・・・よう」也可以单独使用,表示“好像・・・”。

\vspace{10pt}

两者结合使用时,表示“觉得好像・・・”,即“好像・・・”,且不确定度更高。

\vspace{10pt}

\textbf{「よう」的接续规则也适用于此语法}。其中,前续名词否定形,应是「名词 + \textbf{ではない}ような気がする」。
\end{callout}

\begin{example}
    \begin{itemize}
        \item 彼女は最近、\textbf{元気がないよう}な気がする。
        \item この地図は、少し\textbf{複雑なよう}な気がする。
        \item 悪いのは\textbf{私のよう}な気がする。
        \item これは\textbf{彼の字ではないよう}な気がする。
    \end{itemize}
\end{example}


\separatorline

\subsection{「ないうちに」表在・・・之前}

\begin{codeblock}
[・・・动词ない] うちに、 ~
\end{codeblock}

表示\textbf{在某事还没发生之前,做某事}。有\textbf{“趁着・・・”}的意思。

\begin{example}
    \begin{itemize}
        \item お客さんが\textbf{来ないうちに}、掃除をしてしまいましょう。
        
        \textbf{赶在}客人\textbf{来之前},把打扫干净吧。

        \item 暗く\textbf{ならないうちに}、帰りましょう。
    \end{itemize}
\end{example}

\separatorline

\newpage

\subsection{「たきり」表・・・之后未出现预期某事}

\begin{codeblock}
[・・・动词た] きり、 ~
\end{codeblock}

表示动词所表示的\textbf{行为完成后,未出现所预期的某事}。「きり」本身表示的是一种界限,因此,表达的是\textbf{“在此之后”}的状态。
口语中,有时「たきり」读作「たっきり」。

\begin{callout}
    「たきり」后面通常接否定形式,表示“・・・之后就没有・・・了”。

    有时也接肯定形式,表示“・・・之后就一直・・・”。

    \vspace{10pt}

    \textbf{无论接肯定还是否定,都含有“不符合预期”的意思。}
\end{callout}

\begin{example}
    \begin{itemize}
        \item 一度手紙を\textbf{送ったきり}、もう一年も何も連絡していません。
        
        寄了一次信\textbf{之后}(以此为界限),已经一年多没有任何联系了\textbf{(不是说话人所期望的)}。

        \item 彼はアメリカへ\textbf{行ったきり}、帰って来ない。
    \end{itemize}
\end{example}

使用「・・・たきりだ」的形式,可以表示\textbf{至今一直}没有进行所预期的行为。

\begin{example}
    \begin{itemize}
        \item 彼とは卒業式の日に\textbf{会ったきりだ}。
        
        我和他自从毕业典礼那天见过面之后,然后至今没有进行预期的行为。

        即:自从毕业典礼那天见过面之后,就再也没有见过他了。

        \item 祖母は先月風を引いてから、ずっと\textbf{寝たっきりだ}。
        
        我奶奶自从上个月感冒之后,就一直卧病在床,至今没有发生预期的行为(起床活动)。

        即:我奶奶自从上个月感冒之后,就一直卧病在床。
    \end{itemize}
\end{example}

\begin{callout}
    这里,「たきり」\textbf{只表示了至今没有预期的行为},具体至今是什么情况,需要根据上下文来推断。
\end{callout}

\separatorline

\section{词语与用法}

\subsection{「冗談を言う」表开玩笑}

表示“开玩笑,说笑话”。是惯用语,\textbf{「言う」不能替换为其他动词}。

\separatorline

\subsection{「思わず・・・てしまう」表无意识行为}

表示“情不自禁地做了某事”,“不由自主地做了某事”。\textbf{「思わず」是「思わないで」的意思。}

\begin{example}
    \begin{itemize}
        \item 彼の冗談が面白くて、\textbf{思わず笑ってしまった}。
        
        他的笑话很有趣,我\textbf{情不自禁地}笑了出来。

        \item 驚いて、\textbf{思わず}かばんを落としてしまった。
    \end{itemize}
\end{example}
